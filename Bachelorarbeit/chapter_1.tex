\chapter{Einführung}
Heutige elektronisch gesteuerte Geräte versprechen viel Komfort und eine einfache Handhabung im Alltag.
Über das Internet werden die Geräte zunehmend vernetzt und smart.
Dabei sind diese Geräte als Gegenstände erst einmal \textit{Objekte} in der realen Welt, wie Menschen, Tiere, Pflanzen oder Produkte wie Autos oder eben auch Kaffeemaschinen, die von S. Madakam \cite{Madakam2015} unter anderem aufgezählt werden.

Intelligente Dinge (engl. "`Smart Things"') machen unsere Welt nutzerfreundlicher im oben angeführten und versprochenen Sinne.
Sie setzen sich aus einer Gruppe kontrollierbarer und steuerbarer Gegenstände mit einigen Sensorfunktionen zusammen, die an das Internet angebunden sind.
Dies ermöglicht es, jeder Zeit von überall her auf die Produkte zuzugreifen.
Dabei ist der Begriff "`Smart Things"' ein Schlagwort des \ac{IoT}. \cite{Madakam2015}

Unsere Welt benötigt jedoch nicht nur neue und bessere Produkte, sondern auch wiederverwertbare und reparierbare Dinge.
Nun möchte diese Bachelorarbeit eine seit Jahren gut arbeitende Maschine smart machen und sie um eine neue Schnittstelle erweitern.

Daher befasst sich diese Arbeit mit einer eingangs schon erwähnten Kaffeemaschine und in erster Linie mit ihrem Speicher.
Der merkt sich nicht nur Betriebszustände, sondern auch Einstellungen, Zählerstände und evtl. manches mehr.
Die vorhandenen Informationen können nicht nur gelesen, sondern zum Teil auch verändert werden.
Einige weiterführende, smarte Funktionen sind in Maschinen dieser Art bereits ohne eine entsprechende graphische Schnittstelle vorhanden.
Sie müssen nur nutzbar gemacht werden.

A. Kamilaris, A. Pitsillides und V. Trifa stellen in ihrem Paper von 2011 \cite{Kamilaris2011} beispielsweise dar, dass in naher Zukunft Kaffeemaschinen automatisch einen Kaffee nach Benutzerpräferenzen zubereiten können.
Die Möglichkeit, eigene Geräte nun über das Netzwerk zu verwalten, führe zum \textit{Web-enabled Smart Home}.

\section{Aufgabenstellung}
Diese Arbeit reverse engineered ein erhaltenswertes \textit{Objekt} unseres Alltags.
Dafür werden die Funktionen und Abläufe eines \textit{Jura Impressa S9 Kaffeevollautomaten} untersucht.
Das Gerät soll um eine neue Schnittstelle zur Interaktion erweitert werden, nachdem dessen Speicher entschlüsselt ist.
Welches Wort / Byte / Bit speichert welche Information? Welche Bedeutung haben diese für den Betrieb?
Hierfür sollen Skripte erstellt und der Speicher systematisch untersucht werden.

Wenn "`die Denkweise"' der Kaffeemaschine bekannt ist, sollen Werte im \ac{EEPROM} abgefragt, aber auch gezielt verändert sowie Statusinformationen aus dem \ac{RAM} ausgelesen werden.
Die erhaltenen Rohinformationen werden nach den gewonnenen Erkenntnissen aufbereitet und stehen so für weitere Projekte zur Verfügung.

Als praktische Demonstration des Nutzwerts smarter Programmierung, soll am Ende eine Webseite, mit der Profile anlegen werden können, sodass der Nutzer auf Knopfdruck einen Kaffee nach seinen Lieblings-Präferenzen zubereitet bekommt, entstehen.

\section{Aufbau der Arbeit}
Das nächste Kapitel erörtert den Begriff \textit{Reverse Engineering} und zeigt weitere Arbeiten auf.
In Kapitel~\ref{ch:HardwareUndSoftware} werden der reale Kaffeevollautomat, dessen Verkabelung und weitere Abhängigkeiten aufgeführt.
Darauf aufbauend werden in Kapitel~\ref{ch:MethodikUndImplementierung} das strategische Vorgehen und die nötigen Skripte in Form eines größeren C++ Programms entwickelt.
Im Anschluss werden die Ergebnisse in Kapitel~\ref{ch:Ergebnisse} aufbereitet.
Kapitel~\ref{ch:Diskussion} diskutiert aufgetretene Probleme, zeigt die Grenzen dieser Arbeit auf und gliedert sie in das Feld des \textit{Reverse Engineering} ein.
Abschließend fasst Kapitel~\ref{ch:Zusammenfassung} die Arbeit zusammen.
