\chapter{Grundlagen}

% Sektion {Literatur: was gibt es schon}
\section{Begrifflichkeiten}\label{sec:Begrifflichkeiten}
Der Titel dieser Arbeit lautet \texttt{Reverse Engineering eines Kaffeevollautomaten}, aber hinter der Terminologie des Begriffs \textit{Reverse Engineering} steckt ein ganzes Spektrum an Bedeutungen.
E. J. Chikofsky und J. H. Cross differenzieren in ihrem Paper \cite{43044} die Begriffe \textit{Forward Engineering}, \textit{Reverse Engineering}, \textit{Redocumentation}, \textit{Design Recovery}, \textit{Restructuring}, und \textit{Reengineering}.

Zugrunde liegt ein Produkt, welches während seiner Entwicklung mehrere Lebenszyklen durchlaufen hat.
Kassem A. Saleh beschreibt in seinem Buch \cite{Solr-599853700} ausführlicher Entwicklungsaktivitäten, wie die Anforderungsanalyse, das Design, die Implementation, die Tests, die Installation und den Einsatz.
Während der Implementation nennt er Vorgehensmodelle der Softwareentwicklung, wie das Wasserfallmodell, den Prototypenbau, das Spiralmodell, den Objektorientierten Ansatz, das inkrementelle und iterative Modell, sowie das agile Modell.
Jede (Hardware- und) Softwareentwicklung durchläuft dabei, unabhängig vom Modell, die verschiedenen Entwicklungsaktivitäten.
Das voranschreiten zur nächsten Stufe, bis zur Fertigstellung des Produkts, stellt das \textit{Forward Engineering} dar.

Der zweite Begriff des \textit{Reverse Engineering} führt in die Gegenrichtung.
Aus dem implementierten Produkt wird auf das Design, bzw. aus dem Design auf die Spezifikationen geschlossen.
Dabei werden zum einen die Komponenten und ihr Zusammenspiel identifiziert, zum anderen wird aber immer eine höhere Abstraktionsebene, eine vorherige Stufe, rekonstruiert.
Ein wichtiges Zitat, welches in Abschnitt \ref{sec:DiskussionBegriffReverseEngineering}  aufgegriffen wird, lautet: "`Reverse engineering in and of itself does not involve changing the subject system or creating a new system based on the reverse-engineered subject system.
It is a process of examination, not a process of change or replication."'\cite{43044}

\textit{Redocumentation} arbeitet auf einer Ebene und bringt primär eine andere Darstellung.
Das Paper nennt "`dataflow"', "`data structure"' und "`control flow"' als Beispiele, die über Werkzeuge wie Diagramm Generatoren, Syntaxhervorhebung und Querverweis Generatoren erzeugt werden können.
Das Kernziel sei es die Zusammenhänge und Ablaufpfade hervorzuheben.

Breiter angelegt ist das \textit{Design Recovery}, das für die Designwiederherstellung externe Informationen, Schlussfolgerungen, und Unschärfelogik mit einbezieht, um sinnvolle Abstraktionen auf höherer Ebene zu identifizieren, welche nicht aus dem System selbst hätten gewonnen werden können.

\textit{Restructuring} umfasst das Nachbauen einer Darstellung innerhalb einer Ebene.
Funktionalität und Semantik bleiben erhalten, während die Darstellung umgestaltet wird.
Als Beispiel wird die Umstellung von unstrukturiertem Spaghetti Code zu strukturiertem "`goto"' freien Code genannt.
Der Begriff umfasst aber auch Datenmodelle, Entwurfsmuster und Anforderungsstrukturen.
Dabei genügt das Wissen über die Struktur, ohne die Bedeutung dahinter zu verstehen, beispielsweise können "`if"' und "`case"' Ausdrücke ineinander überführt werden, ohne zu verstehen wann welcher Fall eintritt.
Normalerweise werden daher ohne Anpassung der Spezifikation keine Veränderungen vorgenommen.
\textit{Restructuring} ist oft eine Form der präventiven Instandhaltung.

Zuletzt wird der Begriff \textit{Reengineering} oder auch "`Renovation and Reclamation"' als nachträgliche, neu erstellte Form beschrieben.
Dafür geht \textit{Reverse Engineering} zum abstrakten Verständnis dem \textit{Forward Engineering} oder \textit{Restructuring} voraus.
Beim \textit{Reengineering} sind Anpassung der Spezifikation und Veränderungen durchaus möglich.
Ähnlich zum \textit{Restructuring} verändert das \textit{Reengineering} die unterliegende Struktur ohne die Funktionalität zu beeinträchtigen.
Jedoch passiert es selten, dass beim \textit{Reengineering} keine weiteren Funktionalitäten hinzugefügt werden.
Damit ist der Begriff \textit{Reengineering} allgemeiner gefasst als das \textit{Restructuring}.
Aber \textit{Reengineering} ist kein Überbegriff für \textit{Reverse Engineering} und \textit{Forward Engineering}, nur weil es beides beinhaltet.
Beide Disziplinen entwickeln sich unabhängig vom \textit{Reengineering} weiter.

\section{Stand der Technik}
% Sektion (related work)
Eine Bachelorarbeit der Universität Magdeburg zum Thema "`Reverse-engineering a De'Longhi Coffee Maker to precisely bill Coffee Consumption"'\cite{BachelorarbeitDeLonghi} behandelt eine De'Longhi Caffee Maschine.
Das Ziel ist es den Verbrauch exakt zu bestimmen und das System damit um ein Abrechnungssystem zu erweitern.
Dafür wird eine \ac{MCU} über das \ac{SPI} zwischen Master und Slave Einheiten geschaltet und Informationen aus dem proritären De'Longhi Protokoll ausgelesen.
Das Ergebnis der Arbeit ist unter anderem ein Verständnis über das interne Bus Protokoll der Maschine.
% ?! \todo https://www.cl.cam.ac.uk/coffee/qsf/coffee.html ?!

Diese Arbeit hingegen nutzt als Ansatz eine gegebene serielle Schnittstelle des Jura Kaffeevollautomaten mit einem ebenfalls propritären \ac{UART} Protokoll.

\todo\\
Aprilscherz: Hyper Text Coffee Pot Control Protocol (HTCPCP), RFC2324 -> RFC7168 ???\\
\href{http://www.rfc-editor.org/info/rfc2324}{http://www.rfc-editor.org/info/rfc2324}\\
\href{https://www.rfc-editor.org/rfc/rfc2324.txt}{https://www.rfc-editor.org/rfc/rfc2324.txt}\\
\href{https://www.ietf.org/rfc/rfc2324.txt}{https://www.ietf.org/rfc/rfc2324.txt}\\
\href{https://de.wikipedia.org/wiki/Hyper_Text_Coffee_Pot_Control_Protocol}{https://de.wikipedia.org/wiki/Hyper\_Text\_Coffee\_Pot\_Control\_Protocol}\\
\href{https://tools.ietf.org/html/rfc7168}{https://tools.ietf.org/html/rfc7168}\\
