\chapter{Ergebnisse}

\section{Bedeutung der Speicherstellen}




Um den Speicher zu verstehen, klären wir vorab das Rückgabeformat der Speicherauszugs Befehle, sowie den Aufbau des Speichers der Jura Impressa S9.
Dieser Kaffevollautomat besitzt einen \ac{RAM} für die Status, Messwerte und Zwischenberechnungen im Betrieb und einen \ac{EEPROM} für Einstellungen und Zählstände, die auch nach einer Stromunterbrechung erhalten bleiben.

\subsection{EEPROM}
Der \acf{EEPROM} umfasst 512 Bytes.
Über den Befehl \texttt{RT:<address>}, siehe Tabelle~\ref{tbl:kommandos}, lässt sich eine Zeile \ac{EEPROM} Speicher abfragen.
Die Adresse reicht von 0x00 bis 0xF0 in sechzehner Sprüngen.
Als Antwort erhält man hinter dem kleingeschriebenen Kommando eine Zeichenkette bestehend aus 64 Hexadezimalzeichen.
Da immer zwei Hexadezimalzahlen eine Byte (8 Bit mit Werten von 0-255) repräsentieren umfasst eine \ac{EEPROM} Zeile 32 Bytes.
Über den Befehl \texttt{RE:<address>} können direkt Wörter im \ac{EEPROM} abgefragt werden. Die Adresse reicht von 0x00 bis 0xFF.
Die kleinste adressierbare Einheit, das Wort, besteht daher aus zwei Bytes, also vier Hexadezimalzeichen.

\todo Abbildung aus dem Antrittsvortrag: Seite 8/13

\subsection{RAM}
Der \acf{RAM} hingegen umfasst 256 Bytes.
Der Lesebefehl für eine \ac{RAM} Zeile lautet: \texttt{RR:<address>}.
Die Adresse reicht hier von 0x00 bis 0xFF, sodass hier die kleinste Einheit ein Byte, bestehend aus zwei Hexadezimalzeichen, ist.

\todo Abbildung aus dem Antrittsvortrag: Seite 9/13


\section{Aktionsauswirkungen}
Reinigung setzt etliche Zähler zurück; Entnahme der Schale setzt min.
2 Trester zähler zurück; Bits im RAM variieren auch im ausgeschalteten Betriebszustand, einige Informationen nur im eingeschalteten Zustand auslesbar, ander aber auch immer (solange Maschine mit Strom versongt wird)







\todo
Display

Tabelle - verfügbare Displaysymbole!
\label{tbl:Displaysymbole}
