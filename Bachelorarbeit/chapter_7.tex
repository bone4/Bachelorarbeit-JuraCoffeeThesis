\chapter{Zusammenfassung}\label{ch:Zusammenfassung}
Im Rahmen dieser Arbeit wurde der Speicher des Kaffeevollautomaten "`Jura Impressa S9"' reverse engineered.
Dafür wurde eine strukturierte Vorgehensweise in Abschnitt~\ref{sec:Vorgehen} entwickelt.
Einstellung für Einstellung und Funktion für Funktion ergab sich die Bedeutung immer neuer Speicherstellen.
Neu gebündelt wurden diese über eine leicht zugängliche \ac{API} nach außen offen gelegt und abschließend auf einer kleinen Webseite visuell präsentiert.
Dabei wurde die Vorgehensweise und die Aussagekraft der Ergebnisse kritisch hinterfragt und in den Kontext des Reverse Engineering eingeordnet.

\section{Ausblick}
Die in dieser Arbeit aufgeführten Ergebnisse sind noch nicht vollständig, zum Beispiel nennt das Benutzerhandbuch weitere Meldungen wie:\label{FehlendeMeldungen}
\texttt{Gerät verkalkt}, \texttt{Störung 2} oder \texttt{Störung 8}, sowie etliche Menümeldungen während der internen Programmabläufe.
Die Speicherpositionen der Displaymeldungen sind unbekannt, evtl. sogar für alle Sprachen fest im Programmablauf hinterlegt.
Die \ac{RAM} Position für die Wahl der Display Meldung ist aber ebenfalls unbekannt, siehe Abschnitt~\ref{subsec:UnbekannteSpeicherorte}. 
Die verbleibenden drei Meldungen konnten während dieser Arbeit nicht gezielt ausgelöst werden.

Mit entsprechendem zeitlichen und finanziellen Budget kann die Arbeit weiter fortgeführt werden.
Man könnte eine Umgebung schaffen, in der die drei Störungsmeldungen ausgegeben werden um deren Ursprung im \ac{RAM} zu lokalisieren.
Ebenso wäre es interessant fehlende Einheiten zu bestimmen.
Zeit-, Gewichts- oder Durchflusseinheiten würden dem Anwender die Konfiguration vereinfachen.

Es wäre ebenfalls lohnenswert die Firmware aus dem \ac{ROM} der Maschine auszulesen und zu disassemblieren.
Dadurch kann die Erkenntnis, welche Speicherstellen überhaupt vorgesehen und angesprochen werden, erlangt werden.
Ebenso würden Vorgänge bei Funktionsabläufen verständlicher.
% Ghidra (NSA) -> heise online Artikel

Kamilaris, Pitsillides und Trifa nennen in ihrem Paper \cite{Kamilaris2011} als eine Grundanforderung für das intelligente Zuhause den echten Mehrbenutzer Betrieb.
Die Aufbereitung der Webseite könnte durch Crossbar.io\footnote{\url{https://crossbar.io/}} mithilfe von AutobahnC++\footnote{\url{https://github.com/crossbario/autobahn-cpp}} erweitert werden.
Über das \ac{WAMP} könnte regelmäßig der Zustand des Kaffeevollautomaten an alle verbundenen Endgeräte weitergegeben werden, ohne dass ein Endgerät tätig werden muss und während der Kommunikation die Ressource bindet.
