\begin{acronym}[xxxxxxxxx]
  \acro{EEPROM}[EEPROM]{Electrically Erasable Programmable Read-Only Memory}
  \acro{RAM}[RAM]{Ramdom-Access Memory}
  \acro{MCU}[MCU]{Microcontroller Unit}
  \acro{SPI}[SPI]{Serial Peripheral Interface}
  \acro{UART}[UART]{Universal Asynchronous Receiver Transmitter}
  % TODO: IDE aus chapter_3.tex "Monitor der Arduino IDE"
  \acro{API}[API]{Programmierschnittstelle}
  \acro{POSIX}[POSIX]{Portable Operating System Interface}
  \acro{JSON}[JSON]{JavaScript Object Notation}
  \acro{HTML}[HTML]{Hypertext Markup Language}
  \acro{CSS}[CSS]{Cascading Style Sheets}
  \acro{JS}[JS]{JavaScript}
  \acro{AJAX}[AJAX]{Asynchronous JavaScript and XML}
  \acro{CDN}[CDN]{Content Delivery Network}
  \acroplural{API}[APIs]{Programmierschnittstellen}
  
%  \acro{}[]{}
\end{acronym}

%% Markiere diese Akronyme als gesetzt:
%\acused{EEPROM}
%\acused{RAM}
