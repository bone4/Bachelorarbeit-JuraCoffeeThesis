\chapter{Methodik und Implementierung}

\section{Vorgehen}
Die Speicher der Kaffeemaschine, der \ac{EEPROM} und der \ac{RAM}, werden Zeilenweise von einem Programm ausgelesen.
Dieses Programm setzt die einzelnen Informationen zu einem gesamten Speicherauszug zusammen.
Aufeinander folgende Speicherauszüge können dann auf Veränderungen verglichen werden.


Von Hand werden nun Veränderungen angestoßenen, die möglichst elementar sein sollten, um gezielte Aktionen den Wertänderungen bestimmter Speicherzellen zuordnen zu können.
Der dafür nötige Zugriff kann auf zwei Arten erfolgen: direkt am Speicherstein auf der Hauptplanine oder seriell über die vorhandene \ac{UART} Schnittstelle, siehe \ref{subsec:zugangSeriellDirekt}.
Im Folgenden erfolgt die Kommunikation über die \ac{UART} Schnittstelle und mithilfe der \textit{libserial}-Library, siehe \ref{subsec:kommunikationGeraetedateiLibserialLibrary}.

\section{...}

Display: Speicherort unbekannt, über die Befehle (?D0, ?D1xxx, ?D2xxx) systematisch alle ASCII Zeichen gesandt.
Ab dez 128 Darstellung einer Zahl in mehr als einem Kästchen => Speicherüberlauf, nur das normale (nicht erweitere) ASCII alphabet. (war dem wirklich so?, noch einmal testen!)
(Für Kapitel 5: 0-3x und 9x-127 ggf. verpixeltes kanji, aber keine lateinischen Buchstaben oder arabische Zahlen sowie ASCII bekannte Sonderzeichen mehr! Mittelteil ist in der Tabelle dargestellt )

EEPROM: Handbuch einmal durch gegangen und alle Menüoptionen einmal eingestellt, bzw. Schale (Hardware) bewegt.
Bei Skalen immer Minimum, Maximum, Anzahl der Schritte, sowie den Standardwert [N]
Aktionen in vielen Einzelschritten aufgenommen: eine Reinigung => welche Zähler werden zurückgesetzt
Speicher zu einem späten Zeitpunkt gezielt beschrieben
  => Bezugszähler im Menü entschlüsselt: nicht das eine Wort mit dem gleichen Wert, sondern die Summe aus 5+1 Zählern
  => 2 Trester Zähler entdeckt, werden beim Entnehmen der Schale tatsächlich zurückgesetzt
  => Filterwechsel: 500 = 50l => 0,1l Zähleinheiten
  => Reinigung

RAM: mehrere Speicherauszüge in der eingeschalteten Ruheposition (Kaffee bereit) erstellt und regelmäßige Unregelmäßigkeiten festgestellt. Diese Bytes im Kommenden ausgeblendet.
Aktionen ausgelöst, min. 3 Speicherauszüge erstellt und auf Gemeinsamkeiten verglichen.
  Dafür auch Herdare gebaut (s. nkH2 Planinen Foto für Schale)

Für den Schluss
ACHTUNG: während einer Übertragung ist das intere Bus System gehemmt (Display vollzieht keine Änderung), aber auch Gegenrichtung: wechselnde Displaytexte sorgen für Verzögerungen bei der Übertragung
