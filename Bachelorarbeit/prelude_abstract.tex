Im Rahmen dieser Arbeit wurde der Speicher des Kaffeevollautomaten "`Jura Impressa S9"' reverse engineered.
Dafür wurde eine strukturierte Vorgehensweise entwickelt.
Einstellung für Einstellung und Funktion für Funktion ergab sich die Bedeutung immer neuer Speicherstellen.
Neu gebündelt wurden diese über eine leicht zugängliche \acs{API} nach außen offen gelegt und abschließend auf einer kleinen Webseite visuell präsentiert.
Dabei wurde die Vorgehensweise und die Aussagekraft der Ergebnisse kritisch hinterfragt und in den Kontext des Reverse Engineering eingeordnet.
