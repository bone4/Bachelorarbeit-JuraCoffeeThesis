%%
%% DOCUMENT TYPE
%%
\PassOptionsToPackage{figuresright}{rotating} % rotating all sidewaysfigures and -tables to the right (mainly two sided objects on the even left page)

% general options:
% - inputenc        file encoding (should be "utf8" in most cases)
% - de/en           language of your work (influence pre-defined tokens)
% - declaration     adds the mandatory statutory declaration for theses
% - abstract        adds the abstract (from file "prelude_abstract.tex")
% - acknowledgment  adds an acknowledgment (from file "prelude_acknowledgment.tex")
%                   it is a nice gesture to personally thank people who
%                   supported you during your work.
% - symbollist      adds a list of symbols (from file "prelude_symbols.tex")
% - figurelist      adds and automatically creates a list of figures 
% - tablelist       adds and automatically creates a list of tables
% - index           generates an index based on the package "makeidx", please
%                   refer to its documentation for usage on index markup
% - bibbacklinks    adds backlinks from bibliography to the pages, where the
%                   corresponding entry is used (cited)
% - gray            make a gray-style version of the thesis report
%
% PhD thesis specific options
% - cv              adds your cv
% - publishsize     changes the page size from A4 to A5 for print publishing
%                   (please change the font size to 9pt, if you use this option)
% - approved        use this option, after your thesis has been formally approved
%                   (this will change the front page to meet formal/legal requirements)
% - ownpub          adds a second bibliography (from file "ownpub.bib") for your own
%                   publications related to the PhD thesis. According to the latest
%                   examination regulations, own work should be part of the regular
%                   bibliography (this option is hence obsolete)

%\documentclass[de,acronymlist,declaration,bibbacklinks,inputenc=utf8]{tuhhthesis}
\documentclass[de,abstract,acronymlist,declaration,figurelist,tablelist,bibbacklinks,inputenc=utf8]{tuhhthesis}


% for the acronyms
% - footnote        die Langform als Fußnote ausgeben
% - nohyperlinks    wenn hyperref geladen ist, wird die Verlinkung unterbunden
% - printonlyused   nur Abkürzungen auflisten, die tatsächlich verwendet werden.
%    Im printonlyused-Modus kann zusätzlich noch die Option withpage verwendet werden. Hierdurch wird im Abkürzungsverzeichnis zusätzlich die Seitenzahl, auf welcher die Abkürzung als erstes verwendet wurde, ausgegeben.
% - smaller         Text soll kleiner erscheinen, das Paket relsize wird vorausgesetzt
% - dua             es wird immer die Langform ausgegeben
% - nolist          es wird keine Liste mit allen Abkürzungen ausgegeben
\usepackage[printonlyused, withpage]{acronym}

\usepackage{emptypage} % empty backside page on the left at \cleardoublepage

%%
%% SETUP BLOCK
%%

% thesis type, must be one of the following
% - projectwork
% - bachelorthesis
% - masterthesis
% - diplomathesis
% - phdthesis
\setthesistype{bachelorthesis}

% your full name as printed on any official document (e.g., passport)
\author{Niklas Joachim Eberhard Krüger}

% the official title of your work (*must* match the filed title)
\title{Reverse Engineering eines Kaffeevollautomaten}

% the institution of the first examiner (refer to tuhhlangnames.def)
\institute{InstTelematics}

% date of submission as DD.MM.YYYY
\submitdate{22.02.2019}

% your matriculation number (for anything but PhD thesis)
\matrnumber{21491319}

% your course of studies
\course{Informatik-Ingenieurwesen}

% full name and affiliation of first and second examiner
\examinerFirst{Prof. Dr. Volker Turau}{Institut für Telematik\newline Technische Universität Hamburg}
\examinerSecond{Florian Meyer}{Institut für Telematik\newline Technische Universität Hamburg}

\supervisorFirst{Florian Meyer}{Institut für Telematik, Technische Universität Hamburg}
%\supervisorSecond{Volker Turau}{Institute of Telematics, Hamburg University of Technology}

% optional: print the TUB document number on title page
% this only applies, if the document is formally publish under
% a TUB document number
%\tubdoknumber{4711}


% Curriculum Vitae
% only needed for thesis type PhD
%\usepackage[]{currvita}
%\setlength{\cvlabelwidth}{50mm}
%\renewcommand*{\cvlistheadingfont}{\normalfont\sffamily\large\color{tuhh_blue}}
%\renewcommand*{\cvlabelfont}{\normalfont\rmfamily\normalsize\color{tuhh_darkgray}}



%%
%% CONTENT AREA
%%

% mathematical symbols
\input{setup_math}

\newcommand{\todo}{\fcolorbox{black}{yellow}{\textcolor{red}{!!!ToDo!!!}}}

% ~/texmf/tex/latex/tuhh/tuhhcolor.sty
\newcommand{\wort}[1]{\textcolor{tuhh_darkturquoise}{#1$_{16}$}} % EEPROM
\newcommand{\bitTrue}[1]{#1: \wert{$0 \rightarrow 1$}}
\newcommand{\bitFalse}[1]{#1: \wert{$1 \rightarrow 0$}}
\newcommand{\immer}[1]{\TDc\underline{#1}} % RAM Status auch im ausgeschalteten Zustand
\newcommand{\geteilt}[1]{(#1)} % RAM Status Bit mit mehreren Bedeutungen
\newcommand{\bytebit}[2]{Byte \wort{#1} an Bit \textcolor{tuhh_darkturquoise}{#2}} % \bytebit{}{}
\newcommand{\bytebits}[3]{Byte \wort{#1} an den Bits \textcolor{tuhh_darkturquoise}{#2} und \textcolor{tuhh_darkturquoise}{#3}} % \bytebits{}{}{}
\newcommand{\bytebitss}[3]{Byte \wort{#1} an den Bits \textcolor{tuhh_darkturquoise}{#2} bis \textcolor{tuhh_darkturquoise}{#3}} % \bytebitss{}{}{}
\newcommand{\wert}[1]{\textcolor{tuhh_darkred}{#1}}
\newcommand{\bezeichnung}[1]{\textit{#1}}

\renewcommand{\TRhc}[3]{\multicolumn{#1}{#2}{\cellcolor{tuhh_gray}{#3}}} % Row Header for colored rows

\begin{document}

%\phantomsection%prevents "Warning: No destination for bookmark of \addcontentsline"
%\addchap{\tuhh@printTag{ListOfSymbols}}\label{Test}
%\input{prelude_symbols}
%\cleardoublepage

%\renewcommand\contentsname{\tuhh@printTag{LotName}}
%\listoftables
%\addcontentsline{toc}{chapter}{\tuhh@printTag{LotName}}
%\cleardoublepage

% The Chapters
\chapter{Einführung}
Heutige Geräte versprechen viel Komfort und eine einfache Handhabung. Über das Internet werden die Geräte zunehmend vernetzt und smart.
Dabei sind Dinge erst einmal \textit{Objekte} in der realen Welt, wie Menschen, Tiere, Pflanzen, Autos oder eben auch Kaffeemaschinen, die von Madakam \cite{Madakam2015} unter anderem aufgezählt werden.
Intelligente Dinge (engl. "`Smart Things"') machen unsere Welt smart.
Sie setzen sich aus einer Gruppe kontrollierbarer und steuerbarer Dinge mit einigen Sensorfunktionen zusammen, die an das Internet angebunden sind.
Dies ermöglicht es jeder Zeit von überall auf reale Dinge zuzugreifen.
Dabei ist der Begriff "`Smart Things"' ein Schlagwort des \ac{IoT}. \cite{Madakam2015}

Diese Welt benötigt jedoch nicht nur neue und bessere Dinge, sondern auch wiederverwertbare und reparierbare Dinge.
Diese Bachelorarbeit möchte eine seit Jahren gut arbeitende Maschine smart machen und um eine neue Schnittstelle erweitern.

Daher befasst sich diese Arbeit mit einer Kaffeemaschine und in erster Linie mit ihrem Speicher. Dieser merkt sich nicht nur Betriebszustände, sondern auch Einstellungen, Zählerstände und evtl. vieles mehr.
Die vorhandenen Informationen können nicht nur gelesen, sondern zum Teil auch verändert werden.
Moderne Funktionen sind bereits ohne eine entsprechende graphische Schnittstelle in älteren Maschinen vorhanden.
Sie müssen nur nutzbar gemacht werden.

Kamilaris, Pitsillides und Trifa stellen in ihrem Paper von 2011 \cite{Kamilaris2011} dar, dass in naher Zukunft beispielsweise Kaffeemaschinen automatisch einen Kaffee nach Benutzerpräferenzen zubereiten können.
Die Möglichkeit eigene Geräte nun über das Netzwerk verwalten zu können, führe zum \textit{Web-enabled Smart Home}.

\section{Aufgabenstellung}
Diese Arbeit reverse engineered ein \textit{Objekt} unseres Alltags.
Dafür werden die Funktionen und Abläufe eines \textit{Jura Impressa S9 Kaffeevollautomaten} untersucht.
Der Kaffeevollautomat wird um eine neue Schnittstelle zur Interaktion mit dem Gerät erweitert, nachdem dessen Speicher untersucht worden ist.
Welches Wort / Byte / Bit speichert welche Information? Welche Bedeutung haben diese auf den Betrieb?
Hierfür sollen Skripte erstellt werden und der Speicher systematisch untersucht werden.

Wenn die Denkweise der Kaffeemaschine bekannt ist, sollen Werte im \ac{EEPROM} abgefragt, aber auch gezielt verändert werden sowie Statusinformationen aus dem \ac{RAM} ausgelesen werden.
Die erhaltenen Rohinformationen werden nach den gewonnenen Erkenntnissen aufbereitet und stehen so für weitere Projekte zur Verfügung.

Als kleine Demonstration entsteht am Ende ein Programm, welches Profile anlegen kann, sodass jeder Nutzer auf Knopfdruck einen Kaffee nach seinen Lieblings-Präferenzen zubereitet bekommt.

\section{Aufbau der Arbeit}
Kapitel~\ref{ch:Grundlagen} erörtert den Begriff \textit{Reverse Engineering} und zeigt weitere Arbeiten auf.
In Kapitel~\ref{ch:HardwareUndSoftware} werden der reale Kaffeevollautomat, dessen Verkabelung und weitere Abhängigkeiten aufgeführt.
Darauf aufbauend werden in Kapitel~\ref{ch:MethodikUndImplementierung} das strategische Vorgehen und die nötigen Skripte entwickelt.
Im Anschluss werden die Ergebnisse in Kapitel~\ref{ch:Ergebnisse} aufbereitet.
Kapitel~\ref{ch:Diskussion} diskutiert aufgetretene Probleme und zeigt die Grenzen dieser Arbeit auf.
Ebenso wird die Arbeit in das Feld des \textit{Reverse Engineering} eingegliedert.
Abschließend fasst Kapitel~\ref{ch:Zusammenfassung} die Arbeit zusammen.
 % Einführung
\chapter{Grundlagen}

% Sektion {Literatur: was gibt es schon}
\section{Begrifflichkeiten}\label{sec:Begrifflichkeiten}
Der Titel dieser Arbeit lautet \texttt{Reverse Engineering eines Kaffeevollautomaten}, aber hinter der Terminologie des Begriffs \textit{Reverse Engineering} steckt ein ganzes Spektrum an Bedeutungen.
E. J. Chikofsky und J. H. Cross differenzieren in ihrem Paper \cite{43044} die Begriffe \textit{Forward Engineering}, \textit{Reverse Engineering}, \textit{Redocumentation}, \textit{Design Recovery}, \textit{Restructuring}, und \textit{Reengineering}.

Zugrunde liegt ein Produkt, welches während seiner Entwicklung mehrere Lebenszyklen durchlaufen hat.
Kassem A. Saleh beschreibt in seinem Buch \cite{Solr-599853700} ausführlicher Entwicklungsaktivitäten, wie die Anforderungsanalyse, das Design, die Implementation, die Tests, die Installation und den Einsatz.
Während der Implementation nennt er Vorgehensmodelle der Softwareentwicklung, wie das Wasserfallmodell, den Prototypenbau, das Spiralmodell, den Objektorientierten Ansatz, das inkrementelle und iterative Modell, sowie das agile Modell.
Jede (Hardware- und) Softwareentwicklung durchläuft dabei, unabhängig vom Modell, die verschiedenen Entwicklungsaktivitäten.
Das voranschreiten zur nächsten Stufe, bis zur Fertigstellung des Produkts, stellt das \textit{Forward Engineering} dar.

Der zweite Begriff des \textit{Reverse Engineering} führt in die Gegenrichtung.
Aus dem implementierten Produkt wird auf das Design, bzw. aus dem Design auf die Spezifikationen geschlossen.
Dabei werden zum einen die Komponenten und ihr Zusammenspiel identifiziert, zum anderen wird aber immer eine höhere Abstraktionsebene, eine vorherige Stufe, rekonstruiert.
Ein wichtiges Zitat, welches in Abschnitt \ref{sec:DiskussionBegriffReverseEngineering}  aufgegriffen wird, lautet: "`Reverse engineering in and of itself does not involve changing the subject system or creating a new system based on the reverse-engineered subject system.
It is a process of examination, not a process of change or replication."'\cite{43044}

\textit{Redocumentation} arbeitet auf einer Ebene und bringt primär eine andere Darstellung.
Das Paper nennt "`dataflow"', "`data structure"' und "`control flow"' als Beispiele, die über Werkzeuge wie Diagramm Generatoren, Syntaxhervorhebung und Querverweis Generatoren erzeugt werden können.
Das Kernziel sei es die Zusammenhänge und Ablaufpfade hervorzuheben.

Breiter angelegt ist das \textit{Design Recovery}, das für die Designwiederherstellung externe Informationen, Schlussfolgerungen, und Unschärfelogik mit einbezieht, um sinnvolle Abstraktionen auf höherer Ebene zu identifizieren, welche nicht aus dem System selbst hätten gewonnen werden können.

\textit{Restructuring} umfasst das Nachbauen einer Darstellung innerhalb einer Ebene.
Funktionalität und Semantik bleiben erhalten, während die Darstellung umgestaltet wird.
Als Beispiel wird die Umstellung von unstrukturiertem Spaghetti Code zu strukturiertem "`goto"' freien Code genannt.
Der Begriff umfasst aber auch Datenmodelle, Entwurfsmuster und Anforderungsstrukturen.
Dabei genügt das Wissen über die Struktur, ohne die Bedeutung dahinter zu verstehen, beispielsweise können "`if"' und "`case"' Ausdrücke ineinander überführt werden, ohne zu verstehen wann welcher Fall eintritt.
Normalerweise werden daher ohne Anpassung der Spezifikation keine Veränderungen vorgenommen.
\textit{Restructuring} ist oft eine Form der präventiven Instandhaltung.

Zuletzt wird der Begriff \textit{Reengineering} oder auch "`Renovation and Reclamation"' als nachträgliche, neu erstellte Form beschrieben.
Dafür geht \textit{Reverse Engineering} zum abstrakten Verständnis dem \textit{Forward Engineering} oder \textit{Restructuring} voraus.
Beim \textit{Reengineering} sind Anpassung der Spezifikation und Veränderungen durchaus möglich.
Ähnlich zum \textit{Restructuring} verändert das \textit{Reengineering} die unterliegende Struktur ohne die Funktionalität zu beeinträchtigen.
Jedoch passiert es selten, dass beim \textit{Reengineering} keine weiteren Funktionalitäten hinzugefügt werden.
Damit ist der Begriff \textit{Reengineering} allgemeiner gefasst als das \textit{Restructuring}.
Aber \textit{Reengineering} ist kein Überbegriff für \textit{Reverse Engineering} und \textit{Forward Engineering}, nur weil es beides beinhaltet.
Beide Disziplinen entwickeln sich unabhängig vom \textit{Reengineering} weiter.

\section{Stand der Technik}
% Sektion (related work)
Eine Bachelorarbeit der Universität Magdeburg zum Thema "`Reverse-engineering a De'Longhi Coffee Maker to precisely bill Coffee Consumption"'\cite{BachelorarbeitDeLonghi} behandelt eine De'Longhi Caffee Maschine.
Das Ziel ist es den Verbrauch exakt zu bestimmen und das System damit um ein Abrechnungssystem zu erweitern.
Dafür wird eine \ac{MCU} über das \ac{SPI} zwischen Master und Slave Einheiten geschaltet und Informationen aus dem proritären De'Longhi Protokoll ausgelesen.
Das Ergebnis der Arbeit ist unter anderem ein Verständnis über das interne Bus Protokoll der Maschine.
% ?! \todo https://www.cl.cam.ac.uk/coffee/qsf/coffee.html ?!

Diese Arbeit hingegen nutzt als Ansatz eine gegebene serielle Schnittstelle des Jura Kaffeevollautomaten mit einem ebenfalls propritären \ac{UART} Protokoll.

\todo\\
Aprilscherz: Hyper Text Coffee Pot Control Protocol (HTCPCP), RFC2324 -> RFC7168 ???\\
\href{http://www.rfc-editor.org/info/rfc2324}{http://www.rfc-editor.org/info/rfc2324}\\
\href{https://www.rfc-editor.org/rfc/rfc2324.txt}{https://www.rfc-editor.org/rfc/rfc2324.txt}\\
\href{https://www.ietf.org/rfc/rfc2324.txt}{https://www.ietf.org/rfc/rfc2324.txt}\\
\href{https://de.wikipedia.org/wiki/Hyper_Text_Coffee_Pot_Control_Protocol}{https://de.wikipedia.org/wiki/Hyper\_Text\_Coffee\_Pot\_Control\_Protocol}\\
\href{https://tools.ietf.org/html/rfc7168}{https://tools.ietf.org/html/rfc7168}\\
 % Grundlagen
% meine Arbeit deutlich vom gegebenen abgrenzen: Hard-&Software vs. meine Implementierung
\chapter{Hardware und Software}\label{ch:HardwareUndSoftware}
Dieses Kapitel führt die gegebenen und verwendeten Hard- und Software Komponenten in dieser Arbeit auf und schafft die Grundlage für die eigene Implementierung.
Am Ende dieses Kapitels kann der Kaffeevollautomat bereits angesteuert werden.

\section{Der Kaffeevollautomat}
Der "`Jura Impressa S9"', siehe Abbildung~\ref{subfig:Kaffeevollautomat}, ist ein Kaffeevollautomat mit fünf Kaffeebezugstasten: Spezialkaffee, 1 große Tasse Kaffee, 2 große Tassen Kaffee, 1 kleine Tasse Kaffee und 2 kleine Tassen Kaffee.
Auf der rechten Seite befinden sich Bedienelemente für heißes (Tee-)Wasser und Wasserdampf zum Milchaufschäumen.
Der Kaffeevollautomat wird für seinen Betrieb direkt mit der Netzspannung versorgt.

\subfigbox{
  \subfigure[Der ganze Kaffeevollautomat]{\label{subfig:Kaffeevollautomat}\includegraphics[scale=0.2]{images/chapter_3/Jura-Impressa-S9-small}}\hfill%
  \subfigure[Pinbelegung zum Arduino Uno]{\label{subfig:KaffeevollautomatPins}\includegraphics[scale=0.3]{images/chapter_3/Jura-Arduino-Pins-beschriftet-small}}%
}{Der "`Jura Impressa S9"' Kaffeevollautomat}{fig:Kaffeevollautomat}

\subsection{Aufbau und Verkabelung}\label{subsec:AufbauUndVerkabelung}
Hinter der linken Wartungsklappe an der Vorderseite des Kaffeevollautomaten befindet sich neben mehreren Menü-Tasten eine serielle Schnittstelle.
Über ein eigenes \ac{UART} Protokoll kann hierüber mit der Maschine kommuniziert werden.

Abbildung~\ref{subfig:KaffeevollautomatPins} illustriert die Pinbelegung.
Die 5 Volt Leitung kann für ein autark laufendes Projekt genutzt werden.
In dieser Arbeit bezieht der Arduino seine Versorgungsspannung über den am USB Kabel befindlichen Computer.
Von \texttt{TX} nach \texttt{RxD} werden Befehle an den Kaffeevollautomaten verschickt.
Auf der Rückrichtung von \texttt{TxD} nach \texttt{RX} werden Antworten des Kaffeevollautomaten gelesen.
\texttt{GND} ist abschließend die gemeinsame Erdung und Bezugsleitung für die serielle Kommunikation.
Auf der Abbildung kreuzen sich die Leitungen \texttt{RxD} und \texttt{GND} am Stecker auf Seiten des Kaffeevollautomaten.
\texttt{GND} ist über eine schwarze Markierung gekennzeichnet.

Ein Arduino Uno übersetzt als "`Man in the middle"' die bekannten Kommandos in das Format der Kaffeemaschine.
Die serielle Kommunikation läuft über die Pins Nummer 12 und 13.
Über eine weitere serielle Verbindung per USB lässt sich der Arduino ansteuern.

Aus Sicht des Computers ist der Arduino ein Gerätelaufwerk unter \texttt{/dev/ttyACM0} mit einer Baudrate von 9600.
Diese Zahl findet sich zu Beginn des Arduino Uno Skripts wieder.

\subsection{Serielle Kommunikation}
Diese Arbeit baut auf das "`CoffeeMachine"' Projekt~\cite{GitCoffeeMachine} auf und nutzt das Arduino Skript als Grundlage der Kommunikation, ebenso werden Kommandos und erste Speicherstellen aus der Weboberfläche aufgegriffen.
Das Arduino Skript kodiert die \ac{UART} Kommandos von und zu dem Kaffeevollautomaten.

Die Kommandos werden zeichenweise \acs{ASCII} Zeichen für Zeichen übertragen.
Dafür wird ein an den Kaffeevollautomaten adressiertes Byte (ein \acs{ASCII}-Zeichen), bestehend aus acht Bits, in je vier Bytes aufgeteilt.
Die dritte und sechste Stelle der neuen Bytes repräsentieren je zwei Bits des ursprünglichen Bytes.
Bei der Übertragung an den Kaffeevollautomaten werden die restlichen Bits mit Nullen aufgefüllt.
Abbildung~\ref{fig:uart} veranschaulicht dies an dem ersten Byte des Einschaltbefehls, die entsprechende \ac{ASCII}-Kodierung ist in Abbildung~\ref{tbl:Displaysymbole} der zweiten und dritten Spalte zu entnehmen.
Die Kodierung der, von dem Kaffeevollautomaten kommenden, Bytes erfolgt analog.
Laut "`Protocoljura"'\footnote{\url{http://protocoljura.wiki-site.com/index.php/Protocol_to_coffeemaker}} bestehen die irrelevanten Bits der ankommenden Bytes aus einer Null und weiteren fünf Einsen pro Byte.

Vier Bytes kodieren auf diese Weise ein \ac{ASCII} Zeichen und werden fortan als Gruppe bezeichnet.
Zwischen jeder Gruppe gibt es eine Verzögerung von 8ms.
Diese Verzögerung begrenzt hauptsächlich die Übertragungsgeschwindigkeit, was in Abschnitt~\ref{subsec:zugangSeriellDirekt} diskutiert wird.

\begin{figure}
  \begin{center}
    \includegraphics[scale=0.6]{images/chapter_3/UART-Bytes}
    \caption{Umrechnung eines Befehls an den Kaffeevollautomaten}
    \label{fig:uart}
  \end{center}
\end{figure}

\subsection{Kommandos}\label{subsec:Kommandos}
Die Tabelle~\ref{tbl:kommandos} zeigt die Befehlsgruppen, sowie ausgewählte Befehle, die der Kaffeevollautomat versteht.
Diese Befehle stammen aus dem "`CoffeeMachine"' Projekt~\cite{GitCoffeeMachine}.

Befehle beginnen entweder mit zwei Großbuchstaben, gefolgt von einem Doppelpunkt und i.d.R. einer zweistelligen Hexadezimalzahl, oder sie beginnen mit einem Fragezeichen gefolgt von i.d.R. zwei \ac{ASCII}-Zeichen.
Positionsnummern, egal ob Speicherposition, \mbox{Betriebszustands-,} Bezugstasten- oder Steuerungskomponenten-Nummer, werden durch zweistellige Hexadezimalzahlen repräsentiert und reichen von $0_{16}=0_{10}$ bis FF$_{16}=255_{10}$.

Zwei Ausnahmen des ersten Typs sind z.B. \texttt{TY:} zum Abfragen des Maschinen Typs und \texttt{WE:00,01FF} zum Schreiben des Wertes $01$FF$_{16} = 511_{10}$ an die Position $00$.
Eine Ausnahme des zweiten Typs ist z.B. \texttt{?D1DISPLAY}\textvisiblespace\ und \texttt{?D2}\textvisiblespace\textvisiblespace\texttt{TEST}\textvisiblespace\textvisiblespace.
Der volle Umfang möglicher Displayzeichen, sowie deren Benutzung, wird in Abschnitt~\ref{sec:Display} ausgeführt.

Einige Befehle, die aus weiteren Projekten mit Maschinen der S-Reihe bekannt sind, sind in dieser "`Jura Impressa S9"' leider nicht implementiert.
Dazu zählen \texttt{IC:} zum Auslesen aller Eingaben, \texttt{FA:0A} eine unbelegte Bezugstaste, \texttt{CM:} für weitere Status Informationen, \texttt{CS:} für Sensor Informationen und \texttt{PM:} ein Befehl um Musik abzuspielen.
Weitere Befehlsgruppen werden in der \texttt{README.rst} eines Github Repositories aufgeführt\footnote{\url{https://github.com/PromyLOPh/juramote}}.

Die jetzt vorhandene Umgebung kann bereits über den seriellen Monitor der Arduino \ac{IDE} genutzt werden.

\subsection{Speicher}
Über die Befehle \texttt{RE:<address>}, \texttt{RT:<address>} und \texttt{RR:<address>} sind Lesebefehle zu zwei Speichereinheiten des Kaffeevollautomaten bekannt: dem \acf{EEPROM} und dem \acf{RAM}.

\subsubsection{EEPROM}\label{subsubsec:SpeicherDesKaffeevollautomatenEEPROM}
Über den Befehl \texttt{RT:<address>}, siehe Tabelle~\ref{tbl:kommandos}, lässt sich eine Zeile \ac{EEPROM} Speicher abfragen.
Die Adresse reicht von \texttt{0x00} bis \texttt{0xF0} in sechzehner Sprüngen.\footnote{Man ist daran aber nicht gebunden und kann wie im Folgenden beim \ac{RAM} erklärt die Adressen von \texttt{0x00} bis \texttt{0xFF} nutzen. Dies wird in dieser Arbeit auf den \ac{EEPROM} aber \textbf{nicht angewendet}.}
Als Antwort erhält man hinter dem kleingeschriebenen Kommando eine Zeichenkette bestehend aus 64 Hexadezimalzahlen.
Da immer zwei Hexadezimalzahlen ein Byte\footnote{Ein Byte besteht aus 8 Bits mit einem Wertebereich von $0_{10}-255_{10} = 00_{16}-\text{FF}_{16} = $ 0x00 -- 0xFF.} repräsentieren, umfasst eine \ac{EEPROM} Zeile \textbf{32} Bytes.
Der gesamte \ac{EEPROM} umfasst damit insgesamt \textbf{512} Bytes.

Über den Befehl \texttt{RE:<address>} können direkt Einheiten im \ac{EEPROM} abgefragt werden. Die Adresse reicht von \texttt{0x00} bis \texttt{0xFF}.
Die kleinste adressierbare Einheit, \underline{das Wort}, besteht daher aus vier Hexadezimalzahlen, also \textbf{2} Bytes\footnote{Zwei Bytes bestehen aus 16 Bits mit einem Wertebereich von $0_{10}-65\:535_{10} = 00_{16}-\text{FFFF}_{16}$.}.\\
Abbildung~\ref{subfig:EEPROM} zeigt das Speicherschema des \ac{EEPROM}.

\subsubsection{RAM}\label{subsubsec:SpeicherDesKaffeevollautomatenRAM}
Der Lesebefehl für eine \ac{RAM} Zeile lautet: \texttt{RR:<address>}.
Man ist aber nicht gezwungen, am Zeilenanfang zu starten.
Die Adresse reicht hier von \texttt{0x00} bis \texttt{0xFF}.
Als Antwort erhält man hinter dem kleingeschriebenen Kommando eine Zeichenkette bestehend aus 32 Hexadezimalzahlen, also \textbf{16} Bytes.
Bei einer Adresse größer als \texttt{0xF0} gibt es einen Zählerüberlauf und es werden wieder die ersten Bytes zurück gegeben.
Der gesamte \ac{RAM} Speicher verfügt, ähnlich zu \texttt{RT} im \ac{EEPROM}, über sechzehn Zeilen mit je 16 Bytes, also insgesamt \textbf{256} Bytes.
Daraus ergibt sich, dass die kleinste Einheit, ein Byte, aus zwei Hexadezimalzahlen besteht.\\
Abbildung~\ref{subfig:RAM} zeigt das Speicherschema des \ac{RAM}.



\section{Libraries und Frameworks}
Diese Arbeit nutzt "`libraries"' und "`frameworks"'.
"`Libraries"' sind thematisch gebündelte Funktionen und Routinen. Sie bieten eine \ac{API}, die das eigene Programm ansprechen kann.

"`Frameworks"' bieten daneben ganze Programmiergerüste und Routinen. Sie rufen, wenn nötig, selbst vorgesehene Funktionen auf und folgen damit dem Paradigma des "`Inversion of Control"'.

Auf dem aktuellen Stand bieten beide nicht nur leicht zugängliche Aufrufe, sondern sind auch sicher und robust in ihrem Einsatzgebiet.

\subsection{Serielle Kommunikation}
Bei dem Versuch die Grätedatei direkt anzusprechen kam es zu Problemen, die in Abschnitt~\ref{subsec:kommunikationGeraetedateiLibserialLibrary} erörtert werden. Für die zuverlässige serielle Kommunikation kommt daher eine geeignete Library zum Einsatz.

\subsubsection{libserial}
Das selbst entwickelte C++ Programm nutzt "`liberial"'\footnote{\url{https://github.com/crayzeewulf/libserial}} um sicher und zuverlässig über das Gerätelaufwerk mit dem Arduino (und letztlich dem Kaffeevollautomaten) zu kommunizieren.
Diese Bibliothek bietet eine objektorientierte Schnittstelle für alle \ac{POSIX} Systeme und ist unter Linux über die Paketverwaltung installierbar.

\subsection{Speicher- und Austauschformat}
Sowohl bei der Untersuchung des Speichers, als auch am Ende für das aufbereitete Ergebnis, werden Informationen gesammelt, verglichen und öffentlich zugänglich gemacht.
Als beliebtes und flexibles Speicherformat wird in dieser Arbeit \ac{JSON} verwendet.
Es bietet viele Datenformate und ist unbegrenzt verschachtelbar.
In dieser Arbeit werden hauptsächlich Zeichenketten, Zahlen, Unter-Objekte und -Arrays verwendet.
Darüber hinaus kann es leicht vom C++ Programm, der Webseite, oder auch für viele weitere Projekte genutzt werden.

Kamilaris, Pitsillides und Trifa nennen in ihrem Paper \cite{Kamilaris2011} alternativ noch \ac{XML} als oft verwendetes, aber recht langatmiges maschinenlesbares Dateiformat.

\subsubsection{libjsoncpp}
Die "`libjsoncpp"'\footnote{\url{https://en.wikibooks.org/wiki/JsonCpp}} bietet dem C++ Programm eine Lese- und Syntaxanalyse-Funktion zum Aufnehmen eines \ac{JSON} Ausdrucks und eine Möglichkeit, ein \ac{JSON} Objekt kompakt zusammengefasst oder leserfreundlich aufgefächert in einen Ausgabestrom zu schreiben.
Als Ausgabestrom sind reine \ac{JSON} Dateien nach einer Speicherauszugs-Aufnahme oder die Standardausgabe im Gebrauch als API vorstellbar.


\subsection{Webseite}
Eine kleine Webseite soll am Ende die ausgelesenen Werte visuell anschaulich präsentieren.
Die JavaScript-Bibliothek jQuery und das Framework Bootstrap helfen dabei, dies zügig und ansprechend umzusetzen.

\subsubsection{Bootstrap}
Bootstrap\footnote{\url{https://getbootstrap.com/}} ist ein von den Twitter Entwicklern begonnenes Projekt, um ursprünglich intern die Verwaltungswerkzeuge zu vereinheitlichen.
Daraus ist ein ganzes System an Design-Elementen geworden, welches heute sehr populär ist.
\ac{HTML}, \ac{CSS} und \ac{JS} werden zusammen eingesetzt und bieten Entwicklern eine Gitteranordnung für eine Mobile- und die Desktop-Ansicht, Inhaltselemente wie Tabellen und Abbildungen oder auch einzelne Komponenten wie Steckkarten, Prozentanzeigen und Knöpfe.

\subsubsection{jQuery}
jQuery\footnote{\url{https://jquery.com/}} ist eine freie \ac{JS}-Bibliothek, die in der "`slim"'-Variante bereits über Bootstrap eingebunden ist.
Um später wirklich mit dem Kaffeevollautomaten interagieren zu können, wird unter anderem \ac{AJAX} aus dem vollständigen jQuery Paket benötigt (auch wenn daraus später ein synchrones JavaScript mit \acs{JSON} wird).

Über das \ac{CDN} eingebundene Paket stehen ab jetzt einfache \aclp{API} zum Modifizieren der \acs{HTML} Elemente, ein erweitertes Event-System, sowie \acs{AJAX}-Funktionalitäten bereit.

\subsubsection{JavaScript Cookie}
JavaScript Cookie\footnote{\url{https://github.com/js-cookie/js-cookie/}} ist eine kleine \acl{JS}-\ac{API}.
Sie vereinfacht es, Webbrowser-Kekse (Cookies) zum Ablegen persönlicher Einstellungen zu erzeugen, zu bearbeiten und zu entfernen.

\subsubsection{Intro.js}
Zu guter Letzt folgt mit Into.js\footnote{\url{https://introjs.com/}} noch eine Schritt-für-Schritt-Anleitung, die dem Seitenbenutzer eine kurze Einführung in den Aufbau und die Bedienung der Webseite geben soll.
 % Hardware und Software
\chapter{Methodik und Implementierung}\label{ch:MethodikUndImplementierung}
Dieses Kapitel führt das Vorgehen und die Implementierung eines C++ Programms aus.
Dadurch wird es möglich, Aussagen über den Speicher des Kaffeevollautomaten zu treffen.
Immer wiederkehrende Arbeitsabläufe werden von diesem Programm übernommen und auf den unteren Ebenen automatisiert.

\section{Vorgehen}\label{sec:Vorgehen}
Die Speicher des Kaffeevollautomaten, der \ac{EEPROM} und der \ac{RAM}, können zeilenweise ausgelesen werden.
Der nötige Zugriff zum Auslesen kann auf zwei Arten erfolgen: direkt am Speicherstein auf der Hauptplatine oder seriell über die vorhandene \ac{UART} Schnittstelle.
Abschnitt~\ref{subsec:zugangSeriellDirekt} diskutiert die Vor- und Nachteile, die dazu geführt haben im Folgenden die Kommunikation über die \ac{UART} Schnittstelle und mithilfe der \textit{libserial}-Library erfolgen zu lassen.
Dadurch kann die Speicherabfrage als Blackbox Modell, ähnlich zum Blackbox Testing in \cite{Solr-599853700}, betrachtet werden, wo nur mit Ein- und Ausgaben des Kaffeevollautomaten gearbeitet wird, ohne dessen Programm zu kennen.

Unabhängig voneinander wurden beide Speicher untersucht.
Ein eigens entwickeltes C++ Programm setzt die 16 Zeilen einer Speicherabfrage zu einem gesamten Speicherauszug zusammen.
Intern werden dabei je zwei Hexadezimalzahlen in eine ganzzahlige Dezimalzahl umgerechnet und in einem Vektor abgelegt.
Dadurch sind Position und Wert bekannt.

\begin{figure}
  \begin{center}
    \includegraphics[scale=0.6]{images/chapter_4/workflow}
    \caption{Arbeitsablauf zur systematischen Untersuchung des EEPROMs und RAMs}
    \label{fig:workflow}
  \end{center}
\end{figure}

Abbildung~\ref{fig:workflow} visualisiert den Arbeitsablauf, der zur Bestimmung der Speicherstellen in dieser Arbeit angewandt wurde.
Nach dem ersten Schritt ganz links wird nun von Hand eine möglichst elementare Veränderung an dem Kaffeevollautomaten vorgenommen, um gezielten Aktionen die Werteänderungen bestimmter Speicherzellen zuordnen zu können.
Die angewandten Veränderungen werden in Abschnitt~\ref{subsec:AenderungenAnDerMaschine} beschrieben.

Nach einer erneuten Aufnahme eines Speicherauszugs können diese beiden nun verglichen werden.
Das Programm iteriert durch den Vektor und vergleicht die Speicherwerte an der gleichen Position.
Bei Ungleichheit existiert ein Unterschied, der festgehalten wird.
Dieser Ablauf wurde auf alle Einstellmöglichkeiten und Funktionen des Kaffeevollautomaten angewandt.

Bei dem \ac{RAM}, im Gegensatz zum \ac{EEPROM}, fiel jedoch auf, dass sich mehrere Speicherwerte auch ohne eine vorgenommene Veränderung änderten.
Deshalb wurden viele Speicherauszüge im Ruhezustand aufgenommen und die sich regelmäßig ändernden Bytes ausgeschlossen.
Zu Beginn der Ergebnisse über den \ac{RAM} in Abschnitt~\ref{sec:ErgebnisseRAM} sind diese aufgeführt.
Dennoch gab es gelegentliche Unregelmäßigkeiten, sodass das Vorgehen für den \ac{RAM} um eine geschlossene Rückrichtung ergänzt wurde.
In der Regel wurden pro \ac{RAM} Funktion $n=3$ Durchläufe vorgenommen und die gemeinsame Schnittmenge der Veränderungen bestimmt.

Bei einem zweiten Vorgehen wurde gezielt der \ac{EEPROM} an bekannten Speicherstellen beschrieben.
Hintergrund war die unbekannte Position bzw. später die Zusammensetzung des Bezüge-Zählers im Einstellungsmenü des Kaffeevollautomaten.
Abschnitt~\ref{subsec:Vorgehen2} führt dies im Folgenden aus.

\subsection{Veränderungen zur Bestimmung der Speicherstellen}\label{subsec:AenderungenAnDerMaschine}
\subfigbox{
  \subfigure[Dargestellt ist die Fassung im Kaffeevollautomaten.]{\label{subfig:fassung}\includegraphics[scale=0.2]{images/chapter_4/Schale-Fassung}}\\%
  \subfigure[Zu sehen ist die Oberseite des Ersatzes für die Schale.]{\label{subfig:oberseite}\includegraphics[scale=0.29]{images/chapter_4/Schale-Oberseite}}\\%
  \subfigure[Zu sehen ist die Unterseite des Ersatzes für die Schale.]{\label{subfig:unterseite}\includegraphics[scale=0.25]{images/chapter_4/Schale-Unterseite}}%
}{Ersatz der Schale des Kaffeevollautomaten}{fig:ersatzschale}

Für den \ac{EEPROM} war es hilfreich, systematisch durch das Handbuch der Maschine zu gehen und zum Beispiel immer eine Menüoption zu variieren.
Bei einer Skala (wie z.B. beim Kaffeepulver) hat es ausgereicht, den niedrigsten, den höchsten und den Standard-Wert im Speicher zu bestimmen;
deckte sich die einstellbare Anzahl an Abstufungen mit der Differenz der Werte im Speicher, ließ sich auf die verbleibenden Werte schließen.
Auch im normalen Gebrauch fiel unter anderem ein Einschaltzähler im \ac{EEPROM} auf.

Für den \ac{RAM} ging es mit den gezielt auslösbaren Statusmeldungen los.
Das Anheben des Wassertanks oder das Abziehen der unteren Schale löste eine Warnmeldung auf dem Display aus und wurde auch im \ac{RAM} vermerkt.
Ein zu hoher Wasserstand in der Schale konnte ohne Wasser mit einer eigenen Platine vorgegaukelt werden.
Abbildung~\ref{subfig:fassung} zeigt die Fassung an der Rückwand innerhalb des Kaffeevollautomaten für die Schale.
Abbildung~\ref{subfig:oberseite} und \ref{subfig:unterseite} visualisieren die Platine, an der leicht von außen mittels Krokodilklemmen die Kontakte im Inneren kurzgeschlossen werden konnten.
Am schwierigsten sind komplexe Abläufe, wie die Zubereitung eines Kaffees.
Hier konnte nur mehrfach zu gezielt gleichen oder ungleichen Zeitabständen der zweite Speicherauszug angestoßen werden.

Abschließend war auch für das Display ein kleines Programm nötig, um die volle Funktionsvielfalt festzustellen.
Abschnitt~\ref{subsec:tools} führt dies später aus.

In Abschnitt~\ref{sec:AussagekraftDerErgebnisse} wird die Aussagekraft dieser Ergebnisse noch thematisiert.

\subsection{EEPROM gezielt beschreiben}\label{subsec:Vorgehen2}
Ziel war es die unbekannte Position bzw. später die Zusammensetzung des Bezüge-Zählers im Einstellungsmenü des Kaffeevollautomaten zu entziffern.
Dafür wurde die angezeigt Zahl in eine Hexadezimalzahl umgerechnet und im Speicherauszug des \ac{EEPROM} des Kaffeevollautomaten gesucht.
Da die Zahl eine Stromunterbrechung überstand und eine Änderung des einzigen übereinstimmenden Wertependants in Wort \wort{15} keine Änderung der Ausgabe hervorrief, begann die Suche an weiteren bekannten Speicherstellen im \ac{EEPROM}.
Bekannte Zähler in Wörtern wie \wort{00}, \wort{01}, \wort{02}, \wort{0D}, \wort{0E} oder mehreren weiteren wurden einzeln auf sehr hohe Werte geändert.
Manche Änderungen veränderten den Bezüge-Zähler oder lösten Warnmeldungen an dem Kaffeevollautomaten aus.

Es folgte die Erkenntnis, dass sich der Bezüge-Zähler aus mehreren Zählern zusammen setzt.
Der Trester Füllstand in der Schale wird nicht gemessen, sondern im Betrieb gezählt, um den Füllstand zu erfassen.
Auch die Reinigungsankündigung beruht auf Zählerständen über Kaffeebezüge und Spülungen.
Die Wasser-Durchflussmenge des Filters wird ebenfalls erfasst und ab einem festen Wert eine Warnmeldung zum Wechseln ausgegeben.

Das gezielte Eingreifen in die Zählerstände ließ es zu, diese Grenzwerte exakt zu bestimmen.
Die Ergebnisse befinden sich in Abschnitt~\ref{subsec:ErgebnisKaffeezubereitung}.

\section{Das C++ Programm "`./JuraCoffeeMemory"'}
Für diese Arbeit wurde ein C++ Programm entwickelt, das die Kommunikation und die Aufschlüsselung der Antworten übernommen hat.

\subsection{Makefile}
In dem Projekt Ordner befinden sich auf der Hauptebene mehrere \texttt{CPP}- und \texttt{HPP}-, eine \texttt{H}- sowie eine \texttt{Makefile}-Datei.
Sind die in der \texttt{Readme.md} genannten Abhängigkeiten und Voraussetzungen erfüllt, kann das Projekt mit dem Befehl \texttt{make} ohne Zusatzangaben kompiliert werden.
Am Ende sollte dann auf der Hauptebene eine ausführbare Datei namens \texttt{./JuraCoffeeMemory} herauskommen.
Es werden neben Objekt-Dateien auch noch einige weitere Tools im Unterordner \texttt{./tools/} erstellt, auf die später in Abschnitt~\ref{subsec:tools} eingegangen wird.

Der Befehl \texttt{make clean} entfernt die Objekt-Dateien; mittels \texttt{make dist-clean} werden ebenfalls die ausführbaren Programme entfernt und der Projekt-Ordner wird in seinen Ursprungszustand zurück versetzt.

\subsection{Interaktives Menü}
Startet man in einem Terminal das Hauptprogramm \texttt{./JuraCoffeeMemory} wird das Hauptmenü angezeigt. Abbildung~\ref{subfig:terminal1-MainMenu} visualisiert diesen Zustand.
Das Listing~\ref{lst:menu-tree} zeigt schematisch alle Menüoptionen.

\subfigbox{
  \subfigure[Hauptmenü]{\label{subfig:terminal1-MainMenu}\includegraphics[scale=0.4]{images/chapter_4/JuraCoffeeMemory-1-MainMenu}}\hfill%
  \subfigure[EEPROM Skript]{\label{subfig:terminal3-EEPROM-NothingChanged}\includegraphics[scale=0.4]{images/chapter_4/JuraCoffeeMemory-3-EEPROM-NothingChanged}}\\%
  \subfigure[RAM Speicherauszug]{\label{subfig:terminal5-RAM-dump}\includegraphics[scale=0.4]{images/chapter_4/JuraCoffeeMemory-5-RAM-dump}}\hfill%
  \subfigure[Optionen $\leadsto$ Log Pfad ändern]{\label{subfig:terminal7-Options-RamLogFilePath}\includegraphics[scale=0.4]{images/chapter_4/JuraCoffeeMemory-7-Options-RamLogFilePath}}\\%
%  \subfigure[Befehl senden]{\label{subfig:terminal4-Send-TY}\includegraphics[scale=0.4]{images/chapter_4/JuraCoffeeMemory-4-Send-TY}}%
  \subfigure[Auswertungen]{\label{subfig:terminal8-AnalyseDumpsMenu}\includegraphics[scale=0.3]{images/chapter_4/JuraCoffeeMemory-8-AnalyseDumpsMenu}}\hfill%
  \subfigure[Speicherauszug einsehen]{\label{subfig:terminal9-AnalyseFilterDump}\includegraphics[scale=0.3]{images/chapter_4/JuraCoffeeMemory-9-AnalyseFilterDump}}%
}{Interaktives Menü des C++ Programms "`./JuraCoffeeMemory"'}{fig:terminal}

\begin{lstlisting}[label=lst:menu-tree,caption={Menü-Baum ./JuraCoffeeMemory}]
Main Menu
|-- 1: EEPROM Skript
|-- 2: Ram Skript
|-- 4: Send a command
|-- 6: Dump EEPROM
|-- 7: Dump RAM
|-- 9: Options
|   |-- 1: Device path (/dev/ttyACM0)
|   |-- 2: EEPROM log file path (data/eeprom.json)
|   |-- 3: Ram log file path (data/ram.json)
|   `-- 4: Back
|-- 0: Analyse existing dumps
|   |-- Analyse Dumps Menu
|   |   |-- data/xxx.json
|   |   |   `-- Dumps
|   |   |-- C: Clear window
|   |   `-- Q: Quit
|   `-- Q: Quit
`-- Q / q / quit / exit: To leave
\end{lstlisting}

\subsubsection{EEPROM / RAM Skript}\label{subsubsec:EEPROM-RAM-Skript}
Über die Nummer \texttt{1} startet aus dem Hauptmenü das \textit{EEPROM Skript}, über die \texttt{2} das \textit{RAM Skript}.
Siehe hierfür Abbildung~\ref{subfig:terminal3-EEPROM-NothingChanged}.
In beiden Fällen wird das Vorgehen in Abschnitt~\ref{sec:Vorgehen} angewendet.
Das Skript erzeugt initial einen Speicherauszug und merkt sich diesen.
Der Anwender kann sich diesen über die Eingabe \texttt{S} ausgeben lassen oder an diesem Punkt über \texttt{Q} das Skript beenden.
Um fortzufahren kann nun eine Aktion vorgenommen oder direkt ein Befehl abgesetzt werden.
Das Skript erstellt im Anschluss einen weiteren Speicherauszug und fragt nach einem Kommentar zu den vorgenommenen Änderungen.
Nach einer Texteingabe vergleicht das Programm die Speicherauszüge und hält Unterschiede in der Standardausgabe und der \ac{JSON} Ausgabedatei fest.
Ebenso werden in der \ac{JSON} Datei jedes Mal beide Speicherauszüge mit Zeitstempel und Kommentar abgelegt.
Das Skript verwirft den vorletzten Speicherauszug und bietet die Fortführung mit dem Anwender Menü an.

\subsubsection{Einen Befehl abschicken}
Über die Nummer \texttt{4} startet aus dem Hauptmenü eine immer wiederkehrende Eingabeaufforderung, in der Befehle aus der Tabelle~\ref{tbl:kommandos} an den Kaffeevollautomaten verschickt werden können.
Hierüber kann schnell ein Funktionstest der Verbindung zum Kaffeevollautomaten vorgenommen werden.
Über \texttt{Q} kehrt der Anwender in das Hauptmenü zurück.

\subsubsection{Speicherauszug}
Über die Nummer \texttt{6} bzw. Nummer \texttt{7} kann ein einzelner Speicherauszug des \ac{EEPROM}s bzw. des \ac{RAM}s erstellt werden, der unverzüglich in der Standardausgabe ausgegeben wird.
Abbildung~\ref{subfig:terminal5-RAM-dump} zeigt einen Speicherauszug des \ac{RAM}s.
Diese Funktion ist ein Teil des \ac{EEPROM} bzw. \ac{RAM} Skripts.
Es fragt aber nicht nach einem Kommentar und fungiert an dieser Stelle rein lesend.
Mithilfe des Hilfsprogramms zum Formatieren eines Speicherauszugs aus Abschnitt~\ref{subsec:formate-dump} kann man den Speicherauszug um Positionsangaben aufwerten und Speicherwerte dadurch kontrollieren.

\subsubsection{Optionen}
Über die Nummer \texttt{9} gelangt der Anwender in das Optionsmenü, wie in Abbildung~\ref{subfig:terminal7-Options-RamLogFilePath} dargestellt.
Es können für den aktuellen Prozess der Pfad zum Gerätelaufwerk des Arduinos und die Ausgabedateien für das \ac{EEPROM} und \ac{RAM} Skript angepasst werden.
Nach der Eingabe eines neuen Pfades wird dieser sofort übernommen und angezeigt.
Über die Nummer \texttt{4} kehrt der Anwender in das Hauptmenü zurück.

\subsubsection{Analyse aufgenommener Speicherauszüge}
In der \ac{JSON} Datei werden im Knoten \texttt{"data"} die Wertänderungen hinter jedem entsprechenden Byte inklusive Kommentar festgehalten.
Nach mehreren Durchläufen des \ac{EEPROM} / \ac{RAM} Skripts können hier schon direkt einige Zugehörigkeiten abgelesen werden.

Das C++ Programm bietet über die Nummer \texttt{0} auch die Option Unterschiede einzelner Aufnahmen über das Skript erneut auszugeben.
Zuerst werden sämtliche \ac{JSON} Dateien im Unterordner \texttt{data/} aufgelistet, siehe Abbildung~\ref{subfig:terminal8-AnalyseDumpsMenu}.
Nach der Wahl einer Datei wird eine Übersicht aller darin enthaltener Speicherauszüge ausgegeben.
Durch die Wahl der entsprechenden Nummer werden der Kommentar und die veränderten Werte pro Byte aufgeschlüsselt.
In Abbildung~\ref{subfig:terminal9-AnalyseFilterDump} ist dies für zwei Speicherauszüge dargestellt, die jeweils nur eine Änderung beinhalten.

\subsubsection{Angaben in der Aufschlüsselung der Veränderungen}
Sowohl im \ac{EEPROM} / \ac{RAM} Skript als auch in der Analyse aufgenommener Speicherauszüge können Veränderungen eingesehen werden.
Das erste Beispiel aus Abbildung~\ref{subfig:terminal9-AnalyseFilterDump} lautet:
\begin{lstlisting}[label=lst:dumpChanges,caption={Beispiel einer Abweichung zweier Speicherauszüge}]
Your choice is: 2018-10-29 09-13-34: turned on
old value: 28  new value:29  at position: 251 / FB  in 2 byte word no.: 125 / 7D
\end{lstlisting}
Diese Ausgabe ist für den \ac{EEPROM} und \ac{RAM} prinzipiell gleich, die relevanten Stellen sind aber leicht verschieden.
Zu Beginn erscheint die Uhrzeit und der bei der Aufnahme eingegebene Kommentar.
Pro Zeile folgt danach eine Änderung.
In diesem Fall änderte sich der Wert von \texttt{28} auf \texttt{29}.
Diese Aufnahme kommt aus dem \ac{EEPROM}, für den die Befehle auf den 2 Byte \textit{Wort} Adressen beruhen.
Möchte man diesen Wert verändern, ist die letzte Spalte mit der hexadezimalen Zahl von Bedeutung, zum Beispiel \texttt{WE:7D,xxxx}.

Für den \ac{RAM} ist die normale Position eines Bytes von Bedeutung, hier würde sich der dezimale Wert \texttt{29} an der Position \texttt{FB} befinden.
Über den Befehl \texttt{RR:FB} bekäme man die \texttt{29} als anfängliche Hexadezimalzahl zurück gegeben.

\subsubsection{Farbcodierungen}
Die verwendeten Farben sollen dem Anwender Orientierung und Struktur bei der Benutzung verschaffen.
Normale Ausgaben und Benutzereingaben erscheinen in weißer Schrift.
Weiße Schrift auf blauem Grund stellt Menü-Überschriften dar.
In blauer Schrift sind Menüs dargestellt.
Grüne Schrift vor der blinkenden Eingabemarke fordert zur Eingabe auf.
Der Text beschreibt valide Eingaben.
Abschließend beschreibt roter Text einen Fehler.
Die aufgetretene Position wird in weißer Schrift auf rotem Grund hervorgehoben.

\subsection{Die API nach außen}
Das Hauptprogramm kann auch über vier Parameter aufgerufen werden. Die Ausgaben sind dann im fehlerfreien Fall farblose \ac{JSON} Zeichenketten oder einfacher Text in die Standardausgabe.

Mögliche Fehlermeldungen sind für Entwickler in der \texttt{Readme.md} dokumentiert.
Die Fehlerausgabe erfolgt wie im interaktiven Menü mit Farben.
Für Entwickler bietet sich daher der Rückgabewert des Programms an.

\paragraph{./JuraCoffeeMemory command}
Von der Standardeingabe wird eine Zeichenkette erwartet, die an den Kaffeevollautomaten weiter gegeben wird.
Mögliche Eingaben sind die Befehle aus der Tabelle~\ref{tbl:kommandos}.
Die Antwort des Kaffeevollautomaten erscheint als einfacher Text.

\paragraph{./JuraCoffeeMemory eepromWrite}
Bei diesem Aufruf wird von der Standardeingabe ein \ac{JSON} Objekt erwartet.
Alle dort genannten Bezeichnungen werden abgearbeitet und die neuen Werte im Speicher des Kaffeevollautomaten hinterlegt.
Auf der obersten Ebene müssen sich aus \texttt{EEPROM\_Status::getEntriesEEPROM()} bekannte Bezeichnungen befinden, die ein Unterobjekt mit der Bezeichnung \texttt{value} sowie je einen ganzzahligen Wert beinhalten.
Listing~\ref{lst:eepromWrite} veranschaulicht dies an einem Beispiel.

Treffen die Bedingungen zu, wird ein Schreib-Befehl an den Kaffeevollautomaten gesandt.
Die Antworten werden in einer Zeichenkette mit der verwendeten Bezeichnung und dem Text "`\#\#\#"' als Abstandshalter festgehalten und auf der Standardausgabe ausgegeben.
Die Antwort auf das Beispiel Listing~\ref{lst:eepromWrite} steht in Listing~\ref{lst:eepromWriteAnswer}.

\begin{lstlisting}[label=lst:eepromWrite,caption={Beispiel einer JSON Eingabe für./JuraCoffeeMemory eepromWrite}]
{
  "powder_quantity_special_coffee": {
    "value": 11
  },
  "water_quantity_special_coffee": {
    "value":  380
  }
}
\end{lstlisting}

\begin{lstlisting}[label=lst:eepromWriteAnswer,caption={Antwort auf das Beispiel der JSON Eingabe}]
ok: powder_quantity_special_coffee###ok: water_quantity_special_coffee###
\end{lstlisting}

\paragraph{./JuraCoffeeMemory eeprom}
Nach ungefähr 5 Sekunden erhält man die aktuellen Einstellungen und Zählerstände des \ac{EEPROM} in einem kompakten \ac{JSON} Objekt.
Im Gegensatz zum EEPROM Skript werden nur die nötigen Speicherstellen abgefragt.
Die Bezeichnung der Speicherstellen, die auch der Name eines \ac{JSON} Eintrags ist, befindet sich mit der zugehörigen Adresse in \texttt{EEPROM\_Status::getEntriesEEPROM()}.
Eine Adresse besteht aus der Position eines Wortes sowie einer oder beiden Bytes.
Von Hand sind dort zum Teil weitere Hintergrundinformationen, wie Standardwerte über die \texttt{[N]}-Taste des Kaffeevollautomaten, minimale und maximale Schranken, der Wert für einen deaktivierten Zustand oder überhaupt zulässige Werte, hinterlegt.

\paragraph{./JuraCoffeeMemory ram}
Ähnlich zum Abfragen des \ac{EEPROM} wird hier aber der \ac{RAM} in ungefähr 3 Sekunden an den entscheidenden Stellen ausgelesen.
Die Assoziation einer Bezeichnung mit ihrer Speicherposition geschieht in \texttt{RAM\_Status::getEntriesRAM()}.
Hier wird ein Byte und ggf. mehrere zusammenhängende Bits als ein Eintrag abgelegt.
Als Ausgabe erfolgt ein kompaktes \ac{JSON} Objekt.

\subsection{Technische Umsetzung}
Abbildung~\ref{fig:storage_inherit_graph} zeigt schematisch den Zusammenhang der C++ Klassen.
Die Hauptdatei ist die \texttt{JuraCoffeeMemory.cpp}, aus der später auch das ausführbare Programm \texttt{./JuraCoffeeMemory} erzeugt wird.
Sie steht ganz oben in Abbildung~\ref{fig:storage_inherit_graph}.
Die \texttt{main}-Funktion fängt die Argumente des Programm-Aufrufs ab und übergibt entsprechend an die \texttt{EEPROM\_Status} bzw. \texttt{RAM\_Status} Klasse für die \ac{API} oder leitet den Anwender selber durch das interaktive Menü.
Die \textit{API}-Box in Abbildung~\ref{fig:storage_inherit_graph} repräsentiert die \ac{API} mit den beiden Klassen.

Beim Aufruf des \ac{EEPROM} Skripts wird die \texttt{eeprom}-Funktion betreten, für den \ac{RAM} analog die \texttt{ram}-Funktion.
Diese Funktionen bauen zu Beginn eine serielle Verbindung zum Arduino auf und blockieren damit alle anderen Prozesse.
Danach wird eine Instanz der gleichnamigen Klasse erzeugt.
Die Klasse ist in der Box namens \textit{memory} in Abbildung~\ref{fig:storage_inherit_graph} aufgeführt.
Der Konstruktor liest dabei den Speicher aus und füllt Zeile für Zeile den \texttt{raw} Vektor in der Elternklasse \texttt{Storage}.
Abschließend veranlasst der Konstruktor die Umrechnung der hexadezimalen Zeichenketten in einen langen Vektor aus einzelnen Zahlen-Werten, die in \texttt{bytes} abgelegt werden.
Die Instanz wird mit dem Präfix \texttt{old} in der Ursprungsfunktion abgespeichert.

Dem Anwender wird, wie in Abschnitt~\ref{subsubsec:EEPROM-RAM-Skript} beschrieben, ein Menü präsentiert.
Im Falle einer Eingabe in Form eines Befehls, wird diese über die serielle Verbindung gesendet und blockierend eine Antwort abgewartet, die dem Anwender ausgegeben wird.
Gemäß dem Vorgehen in Abschnitt~\ref{sec:Vorgehen} wird dann ein weiterer Speicherauszug erstellt und die neue Instanz mit dem Präfix \texttt{new} abgespeichert.
Abbildung~\ref{fig:storage_inherit_graph} veranschaulicht, dass \texttt{JuraCoffeeMemory.cpp} und die Speicherinstanz gemeinsam auf die serielle Verbindung über die \texttt{SerialConnection} Klasse zugreifen können.

Die Klasse \texttt{Storage} bietet zum byteweisen Abgleich eine Methode mit dem Namen \texttt{diffBytesWith()}.
Argumente sind eine weitere Instanz mit der verglichen wird, ein Vektor mit ausgeschlossenen Bytes, ein Merker ob in die \ac{JSON} Datei geschrieben werden soll und ob bereits ein Kommentar vorliegt.
Das \ac{EEPROM} / \ac{RAM} Skript nutzt ausschließlich das erste obligatorische Argument, indem die \texttt{new} Instanz die Methode und im Argument die \texttt{old} Instanz aufruft.
Die \texttt{ram}-Funktion nutzt an dieser Stelle zusätzlich das erste optionale Argument und exkludiert die im Vorgehen in Abschnitt~\ref{sec:Vorgehen} angesprochenen schwankenden Bytes, die in den Ergebnissen in Abschnitt~\ref{sec:ErgebnisseRAM} explizit genannt werden.
Die nachträgliche Analyse wertet die Unterschiede erneut aus, nutzt aber die letzten beiden optionalen Argumente, um ein erneutes Beschreiben der \ac{JSON} Datei mit den bereits vorliegenden Ergebnissen zu unterdrücken.
Der Kommentar als letztes Argument wird in diesem Fall einfach nur ausgegeben, statt den Anwender nach einem Neuen zu fragen.

Zurück in der \texttt{eeprom}- / \texttt{ram}-Funktion der \texttt{JuraCoffeeMemory.cpp} wird die \texttt{old} Instanz freigegeben und die aktuelle \texttt{new} Instanz zur neuen \texttt{old} Instanz umbenannt, damit der Anwender das Verfahren wiederholen kann.

Die Klasse \texttt{JsonFile}, zu sehen in Abbildung~\ref{fig:storage_inherit_graph}, steht zum Einlesen von \ac{JSON}-Dateien und -Zeichenketten, zum Bearbeiten des Objekts und der Ausgabe in einen Ausgabestrom bereit.
Sie wird ebenfalls zur Aufbereitung der \ac{JSON}-Speicherauszüge in eine geordnete Struktur zur Weiterverarbeitung verwendet.

\begin{figure}
  \begin{center}
    \includegraphics[scale=0.74]{images/chapter_4/class-structure}
    \caption{Schematischer Zusammenhang und Gruppierung der C++ Klassen}
    \label{fig:storage_inherit_graph}
  \end{center}
\end{figure}

\subsubsection{Besonderheiten der technischen Umsetzung / Implementierung}
Die \texttt{EEPROM}- und die \texttt{RAM}-Klasse unterscheiden sich durch die Größenangaben zum Speicher und dem zum Abfragen benötigten Kommando in der entsprechenden \texttt{hpp}-Datei.
Vieles wurde in die gemeinsame Elternklasse \texttt{Storage} ausgegliedert.

Die \texttt{SerialConnection} Klasse setzt das \textit{Singleton-Entwurfsmuster} um\footnote{Die \texttt{JsonFile} Klasse folgt ebenfalls dem Singleton-Entwurfsmuster.}.
Es wird automatisch eine Instanz der Klasse erstellt und ist über eine statische Methode abrufbar.
Dies stellt sicher, dass es nur eine serielle Verbindung zur Zeit gibt.
Darüber hinaus wird beim \texttt{connect()} Methodenaufruf eine Dateisperre auf die serielle Gerätedatei erzeugt.
Dies wiederum garantiert, dass nur eine Prozessinstanz des Programms zur Zeit die Verbindung hält und ungestört bleibt.
Bis zum Aufruf von \texttt{disconnect()} bleibt die Sperre aufrecht erhalten.
Zusätzlich prüft die \texttt{connect()} Methode die Übertragung nach dem Verbindungsaufbau mit dem Testkommando \texttt{TY:} und erwartet von dem Kaffeevollautomaten die eingestellte Antwort \texttt{ty:E1300 CAPU 3\textbackslash r}.
Beim Senden und Empfangen übernehmen die entsprechenden Methoden die Handhabung des Wagenrücklaufs und des Zeilenumbruchs am Ende der Kommandos.
Der Anwender oder Entwickler kann dadurch die Kommandos aus Tabelle~\ref{tbl:kommandos} direkt anwenden.

Die Klassen \texttt{EEPROM\_Status} und \texttt{RAM\_Status} besitzen auch eine Besonderheit.
Die gleich lautende \texttt{scan()} Methode fragt nicht nur die nötigen Speicherstellen ab, sondern vermerkt in dem Vektor \texttt{known\_bytes} ebenfalls deren Position.
Wenn die Methode \texttt{getEntriesEEPROM()} bzw. \texttt{getEntriesRAM()} um weitere Speicherstellen ergänzt wird, ist dadurch sichergestellt, dass auch nur tatsächlich abgefragte Speicherstellen und keine konstanten Nullen zurückgegeben werden.
Andernfalls erhält der Entwickler eine Fehlermeldung.
Weitere Einzelheiten sind in der \texttt{Readme.md} im Projektordner festgehalten.

\section{Weitere kleine Tools}\label{subsec:tools}
Im Unterordner \texttt{./tools/} befinden sich nach dem Aufruf von \texttt{make} drei kleine Programme.

\subsection{Formatieren eines Speicherauszugs}\label{subsec:formate-dump}
Ruft man \texttt{./tools/formate-dump} auf, kann man dem Programm eine Zeichenkette aus $1024$ oder $512$ Hexadezimalzeichen übergeben.
Das Programm erkennt an der Größe \ac{EEPROM} und \ac{RAM} Eingaben und wertet den Speicherauszug um Angaben wie die Speicherposition und Dezimal-/Hexadezimalzahlen Formate auf.

\subsection{Display}
Um den verfügbaren Zeichensatz des Displays in der oberen linken Ecke an der Front des Kaffeevollautomaten zu bestimmen, wurde ein kleines Extraprogramm entwickelt.
Anlass waren Sonderzeichen und die deutschen Umlaute, die sich über die gegebenen Befehle (\texttt{?D0}, \texttt{?D1xxx}, \texttt{?D2xxx}) nicht einfach auf das Display bringen ließen.
Das Programm \texttt{./tools/display} iterierte dafür systematisch über die Zahlen von $0$ bis ungefähr $260$ und gab einen Befehl zum Anzeigen der Dezimalzahl sowie des dazugehörigen \ac{ASCII}-Zeichens an den Kaffeevollautomaten.
Dies umfasst das einfache und erweiterte \ac{ASCII} Alphabet sowie einige weitere Zahlen bzw. Zeichen.
Dabei stellte sich nur ein mittlerer Block des einfachen \ac{ASCII}-Alphabets als interessant heraus, der jetzt eingegrenzt von eben diesem Programm durchlaufen wird.
Ein weiteres Programm namens \texttt{./tools/display-screen-saver} veranschaulicht die damit verbundenen Möglichkeiten.

Die gegebenen Befehle über die serielle Verbindung \texttt{?D0}, \texttt{?D1xxx} und \texttt{?D2xxx} verändern nur den Standardtext "`Kaffee bereit"'.
Andere Texte, wie das Programmmenü oder Warnmeldungen, überlagern den Standardtext mit den einprogrammierten Texten aus den vorhandenen Sprachen.

Möchte man die Ausgaben des ersten Programms nachstellen, ist zu beachten, dass der Kaffeevollautomat eingeschaltet sein muss und auch an die erste Displayzeile ein Ausgabebefehl gegangen sein muss, damit die Ausgabe sichtbar wird.
Das Programm verändert dann die zweite Zeile.
Folgende Befehlseingaben werden empfohlen:
\begin{enumerate}
  \item \texttt{cd JuraCoffeeMemory/}
  \item \texttt{make}
  \item \texttt{./JuraCoffeeMemory}
  \item \texttt{4} (Senden eines Kommandos)
  \item \texttt{AN:01} (Maschine einschalten)
  \item \texttt{?D1xxx} (Etwas Text an die erste Displayzeile)
  \item \texttt{<Strg>+D} (Verlassen des Programms)
  \item \texttt{./tools/display}
\end{enumerate}

\section{Die Webseite}

Die Webseite ist zu sehen in Abbildung~\ref{fig:website}.
Nach ein paar Hinweisen kann im oberen Bereich der einzelne Kaffee sowie die gesamte Maschine konfiguriert werden.
Für jede Kaffeeart ist es möglich, sich seine eigene Konfiguration in einem lokalen Webbrowser-Keks (Cookie) abzuspeichern und komfortabel in den Speicher des Kaffeevollautomaten zu schreiben um sich anschließend einen Kaffee zubereiten zu lassen.
Der Keks kann auch wieder entfernt werden oder sich auf die Standardwerte zurücksetzen lassen.
Prinzipiell kann hierüber der ganze bekannte \ac{EEPROM} überschrieben werden.

Im nächsten Absatz sind Statusinformationen aus dem \ac{RAM} einsehbar.
Einige Meldungen werden am Bild des Kaffeevollautomaten visualisiert.
Der Status kann über \texttt{Refresh} aktualisiert werden.
Einfache "`ja"' oder "`nein"' Meldungen werden durch eine grünes "`on"' oder rotes "`off"' präsentiert.
Informationen, die sich über mehrere Bits erstrecken, werden mit ihrem entsprechenden Zahlenwert dargestellt.

Im letzten Abschnitt befinden sich Zählerstände und Prozent-Anzeigen, die verdeutlichen wie weit es noch bis zur nächsten Warnmeldung, wie Trester leeren, Maschine reinigen oder Filter wechseln, ist.

Wenn am Computer der Mauszeiger auf einer Zahl oder Einstellung zum Finger wird, öffnet sich mit einem Klick ein Fenster, in dem der Wert verändert werden kann.
Nach Möglichkeit werden der Standardwert, der Wert zum Deaktivieren und weitere Hinweis-Texte angeboten.

Unten rechts auf der Seite kann über \texttt{Command} ein Kommando aus der Tabelle~\ref{tbl:kommandos} an den Kaffeevollautomaten abgesetzt werden.
Eine Liste bietet viele Vorschläge mit dazugehörigen Bezeichnungen.
Rechts daneben aktualisiert ein Klick auf \texttt{Refresh} die Statusinformationen aus dem \ac{RAM}.
Ganz rechts in der Ecke kann eine Tour durch die Bedienung der Seite gestartet werden.

\begin{figure}
  \begin{center}
    \includegraphics[scale=0.3]{images/chapter_4/Webseite}
    \caption{Webseite zur Kontrolle und Steuerung des Kaffeevollautomaten als Visualisierung der API.}
    \label{fig:website}
  \end{center}
\end{figure}

\subsection{Technische Umsetzung}
Die Webseite befindet sich im Projektunterordner \texttt{./website/}.
Die Datei \texttt{data.php} nimmt Parameter entgegen und ruft damit das C++ Programm mit evtl. Datenübergaben auf.
Die Antwort des C++ Programms wird in einem \ac{JSON}-Objekt verpackt und der Webseite zurück gegeben.
Bei Problemen wird auch die Fehlernummer (der Rückgabewert des Programms) einschließlich der Meldung leicht zugänglich in einem \ac{JSON}-Objekt zurück gegeben.

Die für den Benutzer ersichtliche Webseite besteht primär aus den Dateien \texttt{index.html}, \texttt{script.js} für die Interaktion und \texttt{style.css} für das individuelle Layout.
Weitere eingebundene Dateien sind das Favicon im Unterordner \texttt{favicon/}, Bilder im Unterordner \texttt{img/}, \texttt{intro.min.js} und \texttt{intro.min.css} für die Tour durch die Webseite.
\texttt{offline\_eeprom.json} und \texttt{offline\_ram.json} bieten Beispielwerte für eine Offline-Funktion, in der die Webseite ersichtlich wird, ohne eine Verbindung zu dem Kaffeevollautomaten zu haben.

Das \ac{HTML} Dokument der Webseite besteht aus vielen leeren Inhaltselementen, die im Attribut \texttt{id="..."} Einträge haben, die mit den Bezeichnungen der \ac{JSON} Elemente der Ausgabe von der API des C++ Programms übereinstimmen.
Beim Seitenaufbau wird der \ac{EEPROM} abgefragt.
Wenn die Antwort nach ungefähr 5~Sekunden vorliegt wird der \ac{RAM} abgefragt und anschließend nach ungefähr weiteren 3~Sekunden für jeden \ac{JSON} Eintrag der Wert in das gleichnamige \ac{HTML} Inhaltselement eingetragen.

Eine Ausnahme bildet die Temperatur, die für alle drei Kaffeegrößen gilt, aber im Dokument drei verschiedene ID-Bezeichnungen besitzen muss.
Ebenso ist etwas Handarbeit bei der Ausgabe der Prozent-Anzeigen vonnöten.

Das bei einem Klick aufgehende Fenster zum Bearbeiten von Werten aus dem \ac{EEPROM} arbeitet ebenfalls über die ID-Bezeichnung und entnimmt weitere Informationen aus dem \ac{JSON} Objekt welches zuletzt abgefragt und abgespeichert wurde.
 % Methodik und Implementierung
\chapter{Ergebnisse}

\section{Bedeutung der Speicherstellen}




Um den Speicher zu verstehen, klären wir vorab das Rückgabeformat der Speicherauszugs Befehle, sowie den Aufbau des Speichers der Jura Impressa S9.
Dieser Kaffevollautomat besitzt einen \ac{RAM} für die Status, Messwerte und Zwischenberechnungen im Betrieb und einen \ac{EEPROM} für Einstellungen und Zählstände, die auch nach einer Stromunterbrechung erhalten bleiben.

\subsection{EEPROM}
Der \acf{EEPROM} umfasst 512 Bytes.
Über den Befehl \texttt{RT:<address>}, siehe Tabelle~\ref{tbl:kommandos}, lässt sich eine Zeile \ac{EEPROM} Speicher abfragen.
Die Adresse reicht von 0x00 bis 0xF0 in sechzehner Sprüngen.
Als Antwort erhält man hinter dem kleingeschriebenen Kommando eine Zeichenkette bestehend aus 64 Hexadezimalzeichen.
Da immer zwei Hexadezimalzahlen eine Byte (8 Bit mit Werten von 0-255) repräsentieren umfasst eine \ac{EEPROM} Zeile 32 Bytes.
Über den Befehl \texttt{RE:<address>} können direkt Wörter im \ac{EEPROM} abgefragt werden. Die Adresse reicht von 0x00 bis 0xFF. Die kleinste adressierbare Einheit, das Wort, besteht daher aus zwei Bytes, also vier Hexadezimalzeichen.

\todo Abbildung aus dem Antrittsvortrag: Seite 8/13

\subsection{RAM}
Der \acf{RAM} hingegen umfasst 256 Bytes.
Der Lesebefehl für eine \ac{RAM} Zeile lautet: \texttt{RR:<address>}. Die Adresse reicht hier von 0x00 bis 0xFF, sodass hier die kleinste Einheit ein Byte, bestehend aus zwei Hexadezimalzeichen, ist.

\todo Abbildung aus dem Antrittsvortrag: Seite 9/13


\section{Aktionsauswirkungen}
Reinigung setzt etliche Zähler zurück; Entnahme der Schale setzt min. 2 Trester zähler zurück; Bits im RAM variieren auch im ausgeschalteten Betriebszustand, einige Informationen nur im eingeschalteten Zustand auslesbar, ander aber auch immer (solange Maschine mit Strom versongt wird)
 % Speicher                     % Analyse / Zusammenfassung / Ergebnisse
\chapter{Diskussion}\label{ch:Diskussion}
Dieses Kapitel benennt aufgetretene Probleme während der Arbeit und hinterfragt die Ergebnisse aus dem vorherigen Kapitel.
Auch der Begriff des \textit{Reverse Engineering} wird hier wieder aufgegriffen.

\section{Probleme}
Nicht nur die Kommunikation mit dem Kaffeevollautomaten, sondern auch das Verständnis der Vorgänge in der Maschine erfordern Zeit und Arbeit.

\subsection{Kommunikation mit dem Kaffeevollautomaten} \label{subsec:zugangSeriellDirekt}
Der direkte Zugang ist wahrscheinlich am schnellsten, im laufenden Betrieb aber schlecht möglich, da die Speicheradressen aktiv abgefragt werden müssten und damit Kurzschlüsse in der Elektronik produziert werden würden.
Die serielle \ac{UART} Schnittstelle ermöglicht das Abfragen im laufenden Betrieb, also auch im \ac{RAM}.
Das Auslesen benötigt jedoch seine Zeit, ungefähr 9 Sekunden für den gesamten \ac{RAM} und ungefähr 15 Sekunden für den gesamten \ac{EEPROM}.
Gerade im \ac{RAM} gibt es viele Veränderungen im Ruhezustand, sodass auf diesem Weg nie ein zusammenhängender Speicherauszug zu einem festen Zeitpunkt ausgelesen werden kann.

Ein weiterer Nachteil ist, dass während einer Übertragung das interne Bus System gehemmt ist.
Eingaben an der Maschine, wie die Navigation durch das Menü, werden während der Erstellung eines Speicherauszugs verzögert umgesetzt.
Aber auch in der Gegenrichtung sorgen wechselnde Displaytexte für Verzögerungen während der seriellen Übertragung.

Durch die Optimierung für die \ac{API} benötigt das Auslesen nur noch ungefähr 3 Sekunden für den \ac{RAM} und ungefähr 5 Sekunden für den \ac{EEPROM}.
Abbildung~\ref{fig:API-EEPROM} und Abbildung~\ref{fig:API-RAM} zeigen die dabei abgefragten Speicherstellen.

\subsection{Serielle Kommunikation} \label{subsec:kommunikationGeraetedateiLibserialLibrary}
\paragraph{Gerätedatei direkt ansprechen}
Um mit dem Arduino und letztlich dem Kaffeevollautomaten zu kommunizieren, war der erste Versuch den Dateideskriptor \texttt{/dev/ttyACM0} sowohl lesend als auch schreibend zu nutzen.
Zunächst funktionierte dies sehr gut, bis nach wenigen Tagen kaum reproduzierbares Fehlverhalten auftrat.
Gerade Antworten der Kaffeemaschine kamen nur noch in Bruchstücken an.
Dies wurde bei der Einrichtung bereits im Projekt "`CoffeeMachine"'\cite{GitCoffeeMachine} in der \texttt{Readme.md} für die Ausführung auf einem Raspberry Pi beschrieben.

Versuche, dies direkt zu lösen, indem eine funktionierende Umgebung nachgebaut wurde, schlugen fehl.
Dabei konnte über \texttt{screen /dev/ttyACM0 9600} und \texttt{exit} die serielle Verbindung initiiert werden.
Die Einstellungen können dann über \texttt{stty -F /dev/ttyACM0} eingesehen werden, jedoch hatte es nicht denselben Effekt, nur diese Einstellungen gezielt über den Befehl
\begin{lstlisting}[label=lst:stty,caption={stty zum setzen der Verbindungseinstellungen}]
stty -F /dev/ttyACM0 9600 raw \
ignbrk -brkint -icrnl -imaxbel \
-opost -onlcr \
-isig -icanon -iexten -echo -echoe -echok -echoctl -echoke
\end{lstlisting}
zu setzen.

\paragraph{C++ und die \textit{libserial}-Library}
Für eine sichere Verbindung wird in dem C++ Programm auf die im Linux Repository erhältliche Library \textit{libserial} zurückgegriffen.
Beim Verbindungsaufbau über die \textit{connect()} Methode der \textit{SerialConnection} Klasse wird die Schnittstelle initialisiert.
Hier werden der Gerätepfad, die Baudrate und eine erwartete Mindestlänge festgelegt.

Kommandos gehen zuverlässig an den Arduino raus und ausgelesene Antworten kommen nun im Ganzen an.
Der \textit{read} Befehl ist ein blockierender Aufruf, der das Programm anhält bis eine Antwort vorliegt.
Dies stellt für die Speicherauslesung kein Problem dar, da die ordnungsgemäßen Befehle immer ein \texttt{ok:} oder \texttt{xx:0-F} Speicherauszug zurückgeben.

\subsection{Unbekannte Speicherorte}\label{subsec:UnbekannteSpeicherorte}
Bis auf den Speicherort des veränderbaren Standardtextes konnten alle Einstellungen und Grenzwerte sowie ein paar Funktionsabläufe festen Speicherstellen zugeordnet werden.
Da der Standardtext eine Stromunterbrechung nicht übersteht, liegt er sehr wahrscheinlich im \ac{RAM} oder einem weiteren flüchtigen Speicher der Maschine.
Selbst unter der Annahme, dass die Maschine vielleicht nur die halbe \ac{ASCII}-Tabelle pro Zeichen benutzt und dadurch zwei Displayzeichen pro Byte kodiert, konnten einfache und monotone Texte wie \texttt{AAAAAAAA} nicht in einem Speicherauszug wiedergefunden werden.

Um dies weiter zu ergründen, muss künftig ein Blick in die Maschine erfolgen, um die vorhandenen Speicher und deren Größen zu bestimmen.
Nach dem Datenblatt des Mikrocontrollers \cite{JuraMicrocontroller} gibt es vier Ausführungen, von der aber keine den 256 Byte \ac{RAM} und 512 Byte (256 Wörter) \ac{EEPROM}, die über die seriellen Befehle adressierbar sind, entspricht.

Es wäre vorstellbar, dass die gegebenen Befehle aus Tabelle~\ref{tbl:kommandos} nicht den insgesamt zur Verfügung stehenden Speicher adressieren und nach außen kommunizieren.
Andererseits ist es ebenfalls möglich, dass viele unbekannte Speicherstellen gar keine Funktion besitzen;
der vorhandene Speicher dieses Standard Mikrocontrollers würde von dem Programm dann nicht voll ausgenutzt werden.

\subsection{Weitere Automatisierung durch das C++ Programm}
Für den \ac{EEPROM} ist es ein vertretbarer Aufwand, Einstellungen am Gerät vorzunehmen, das Skript anzustoßen und einen Kommentar zu hinterlegen.
Die Ergebnisse können aus der \ac{JSON} Datei abgelesen werden.

Die wechselhaften Sprünge im \ac{RAM} erforderten ein wiederholtes Anwenden des Skripts, wie in Abschnitt~\ref{sec:Vorgehen} beschrieben.
Hier könnte das C++ Programm dahingehend ausgebaut werden, $n=3$ Durchläufe mit dem selben Kommentar zu versehen.
Die Auswertung sollte aber von Hand erfolgen und es sollten Ergebnisse aus variierenden Umgebungsbedingungen einbezogen werden.
Zum Beispiel den Wasserfilter einmal aktivieren und einmal deaktivieren.
Es bedürfte einer sehr guten künstlichen Intelligenz, diese Umgebungen vorzuschlagen und die Ergebnisse zusammenzuführen.

Mit dem in Abschnitt~\ref{sec:Vorgehen} beschrieben ersten Vorgehen lassen sich Statusinformationen und Einstellmöglichkeiten sehr gut auslesen.
Bedingte Abhängigkeiten, wie die Zusammensetzung eines Bezüge-Zählers aus mehreren Zählern, oder Abläufe, wie die zwei Schritte einer Kaffeezubereitung, wären durch ein starr hochgezogenes Computerprogramm unentdeckt geblieben.

Ausbaufähig ist die wieder bessere Trennung des \ac{EEPROM}s und des \ac{RAM}s.
In den gefundenen Unterschieden des Skripts können weniger relevante Spalten ausgeblendet werden und es könnten die \ac{JSON} Dateien mit einem Merker versehen werden, um der nachträglichen \textit{Aufschlüsselung der Veränderungen} im C++ Programm eine bessere Differenzierung zu ermöglichen.
Zur Zeit muss für den \ac{RAM} von Hand die Zeile 192 der \texttt{JuraCoffeeMemory.cpp} einkommentiert werden.

\section{Aussagekraft der Ergebnisse}\label{sec:AussagekraftDerErgebnisse}
Die \ac{EEPROM} Einstellmöglichkeiten können recht sicher festen Speicherstellen zugeordnet werden.
Bei einigen partiellen Zubereitungszählern ist aber nicht bekannt, wann sie zählen und wofür.
Die Firmware fehlt an dieser Stelle.

Im \ac{RAM} sind mehrere Status Bits einigermaßen aussagekräftig, vor allem aber die im ausgeschalteten Betriebszustand, die auch in Tabelle~\ref{tbl:RAM1} und Tabelle~\ref{tbl:RAM2} hervorgehoben sind.
Durch viele unregelmäßige Veränderungen wurden aber mehrere Bytes außer Acht gelassen, die an bestimmten Bit Positionen wichtige Merker enthalten könnten.

Unsicherheit besteht durch mehrere Speicherpositionen für ein und dieselbe Funktion.
Entweder verbirgt sich hinter einzelnen Positionen noch eine andere Bedeutung oder eine bisher unbekannte Abhängigkeit für weitere Funktionen.
Zustände könnten im \ac{RAM} tatsächlich mehrfach erfasst werden.

\section{Einordnung zur Terminologie des Reverse Engineerings}\label{sec:DiskussionBegriffReverseEngineering}
Diese Arbeit besteht im Wesentlichen aus den zwei Teilen, den Speicher zu analysieren und zu verstehen, bzw. das Wissen darüber in einer neuen Oberfläche einfach zur Verfügung zu stellen.
Ein kurzer Rückblick in Abschnitt~\ref{sec:Begrifflichkeiten} erinnert an die Begriffe \textit{Forward Engineering}, \textit{Reverse Engineering}, \textit{Redocumentation}, \textit{Design Recovery}, \textit{Restructuring} und \textit{Reengineering}.

Für den ersten Teil wurden eine gegebene serielle Schnittstelle und bereits implementierte Befehle ausgenutzt, um den Speicher auszulesen.
Viele Aufnahmen, mit kleinen Veränderungen zwischendurch, ließen Rückschlüsse auf die Bedeutung der Werte an bestimmten Speicherstellen zu.
Gerade mit den Überlegungen, was während einer Aktion am Kaffeevollautomaten alles im Speicher passiert, lässt sich diese Tätigkeit dem Begriff \textit{Reverse Engineering} zuordnen.
Aus dem fertig implementierten und gegebenen System werden Rückschlüsse auf das Design der Maschine gezogen.
\textit{Redocumentation} trifft aufgrund fehlender Spezifikationen und Quellcodeauszügen wenig zu.
Betrachtet man aber das Benutzerhandbuch und Interneteinträge über Erfahrungen mit der Maschine als externe Informationen, sowie Schlussfolgerungen und Unschärfelogik über den \ac{RAM}, kann dieser Part auch als \textit{Design Recovery} über den Speicher angesehen werden.

Mit diesem Wissen des ersten Teils wurde dann im zweiten Teil etwas Neues geschaffen.
Das Abfragen der aktuellen Einstellungen und Zustände bietet eine neue Oberfläche, die als \textit{Redocumentation} oder \textit{Restructuring} angesehen werden kann.
Für die \textit{Redocumentation} spricht das neue Ausgabeformat auf der Ebene der Implementierung.
Ein eigenes Gerät ist zur Alternative (Ausnahmen siehe~\ref{FehlendeMeldungen}) für das kleine Display des Geräts geworden und gibt den aktuellen Zustand in ausführlicherer Weise aus.
Ohne eine präventive Instandhaltung beabsichtigt zu haben ist \textit{Restructuring} aber auch zutreffend, da Wissen aus dem ersten Teil hier eingeflossen ist.
Wenn die Bedeutung bestimmter Speicherstellen nicht bekannt gewesen wäre, hätten die eingestellten Werte keine Bedeutung gehabt, was für den Begriff \textit{Restructuring} spricht.

Für eine individuelle Zubereitung wird nur die Datengrundlage kurz zuvor mit einem Impuls von außen verändert, der Kaffee wird dabei nach wie vor von den Standard Abläufen der Kaffeemaschine zubereitet.
Die neue Oberfläche kann als \textit{Restructuring} oder dem allgemeineren \textit{Reengineering} angesehen werden.
Da die neue Oberfläche über die serielle Schnittstelle in Verbindung mit einem eigenen Endgerät aber sehr von den Bordmitteln der Maschine abweicht, passt vielleicht das \textit{Reengineering} etwas besser.
Die Funktionalitäten der Maschine sind aber immer noch die Gleichen, das Programmmenü muss nun jedoch nicht mehr verwendet werden, um die allgemeinen Geräteeinstellungen für den eigenen Kaffee anzupassen.

Im Detail ist die genaue Zuordnung der Begrifflichkeiten immer eine persönliche Auslegung dieser Begriffe.
Sehr zutreffend ist das Zitat aus Abschnitt~\ref{sec:Begrifflichkeiten}, denn diese Arbeit hat den Speicher intensiv betrachtet und mittels der Ergebnisse aus dem zweiten Teil um neue Perspektiven bereichert, ohne das Ziel verfolgt zu haben, die Maschine zu klonen oder umzubauen.
 % Diskussion
\chapter{Zusammenfassung}\label{ch:Zusammenfassung}
Im Rahmen dieser Arbeit wurde der Speicher des Kaffeevollautomaten "`Jura Impressa S9"' reverse engineered.
Für den Aufbau und die Verkabelung ist auf das Arduino Skript und die Dokumentation aus dem Projekt~\cite{GitCoffeeMachine} zurückgegriffen worden.
Viele der dort gebündelten Informationen stammen wiederum aus der Community\footnote{\url{http://protocoljura.wiki-site.com/}}.
Darauf aufbauend folgte bereits in Kapitel~\ref{ch:HardwareUndSoftware} die Definition der \textit{UART Gruppe} und des Speicheraufbaus mit der Definition des \textit{EEPROM Wortes}.

Zur Untersuchung des Speichers und der darin enthaltenen Informationen zur Überwachung und Steuerung des Kaffeevollautomaten, wurde eine strukturierte Vorgehensweise in Abschnitt~\ref{sec:Vorgehen} entwickelt.
Mit einem C++ Programm sind Speicherauszüge vor und nach einer Veränderung an der Maschine aufgenommen und verglichen worden.
Die Ergebnisse mit den Speicherpositionen bestimmter Einstellungen oder Funktionen listet Kapitel~\ref{ch:Ergebnisse} auf.

Neu gebündelt wurden diese über eine leicht zugängliche \ac{API} nach außen offengelegt.
\ac{EEPROM} und \ac{RAM} können dadurch schneller abgefragt werden.
Die Erstellung eines vollständigen Speicherauszugs bräuchte wesentlich mehr Zeit.
Abschließend wird die \ac{API} auf einer kleinen Webseite visuell präsentiert.
Im Abschnitt~\ref{sec:AussagekraftDerErgebnisse} dieser Arbeit sind die Vorgehensweise und die Aussagekraft der Ergebnisse kritisch hinterfragt.

Das Reverse Engineering besitzt mit seinen benachbarten Disziplinen aus Abschnitt~\ref{sec:Begrifflichkeiten} ein weites Spektrum an Möglichkeiten, alltägliche Gegenstände zu erhalten und weiterzuentwickeln.
In dieser Bachelorarbeit wurde deutlich, dass die Erarbeitung an Wissen über den internen Speicher des Kaffeevollautomaten als \textit{Reverse Engineering} oder \textit{Design Recovery} bezeichnet werden kann.
Die darauf aufbauende \ac{API} kann wiederum als \textit{Redocumentation} oder \textit{Restructuring} bezeichnet werden.
Die Webseite ist schlussendlich als neue Oberfläche der Ertrag des \textit{Reengineering} und zeigt eine smatere Nutzbarkeit des Gerätes.

\section{Ausblick}
Die in dieser Arbeit aufgeführten Ergebnisse sind noch nicht vollständig, zum Beispiel nennt das Benutzerhandbuch weitere Meldungen wie:\label{FehlendeMeldungen}
\texttt{Gerät verkalkt}, \texttt{Störung 2} oder \texttt{Störung 8} sowie etliche Menümeldungen während der internen Programmabläufe.
Die Speicherpositionen der Displaymeldungen sind unbekannt, evtl. sogar für alle Sprachen fest im Programmablauf hinterlegt.
Die \ac{RAM} Position für die Wahl der Display Meldung ist ebenfalls unbekannt, siehe Abschnitt~\ref{subsec:UnbekannteSpeicherorte}. 
Die verbleibenden drei Meldungen konnten während dieser Arbeit nicht gezielt ausgelöst werden.

Mit entsprechendem zeitlichen und finanziellen Budget könnte die Arbeit weiter fortgeführt werden.
Man könnte eine Umgebung schaffen, in der die drei Störungsmeldungen ausgegeben werden, um deren Ursprung im \ac{RAM} zu lokalisieren.
Ebenso wäre es interessant, fehlende Einheiten zu bestimmen.
Zeit-, Gewichts- oder Durchflusseinheiten würden dem Anwender die Konfiguration vereinfachen.

Es wäre ebenfalls lohnenswert, die Firmware aus dem \ac{ROM} der Maschine auszulesen und zu disassemblieren.
Dadurch kann die Erkenntnis, welche Speicherstellen überhaupt vorgesehen und angesprochen werden, erlangt werden.
Ebenso würden Vorgänge bei Funktionsabläufen verständlicher.
% Ghidra (NSA) -> heise online Artikel

A. Kamilaris, A. Pitsillides und V. Trifa nennen in ihrem Paper \cite{Kamilaris2011} als eine Grundanforderung für das intelligente Zuhause den echten Mehrbenutzer Betrieb.
Die Aufbereitung der Webseite könnte durch Crossbar.io\footnote{\url{https://crossbar.io/}} mithilfe von AutobahnC++\footnote{\url{https://github.com/crossbario/autobahn-cpp}} erweitert werden.
Über das \ac{WAMP} könnte regelmäßig der Zustand des Kaffeevollautomaten an alle verbundenen Endgeräte weitergegeben werden, ohne dass ein Endgerät tätig werden muss und während der Kommunikation die Ressource bindet.
 % Zusammenfassung und Ausblick

% Bibliography
% if you have cited papers that are not referenced, but important for your work,
% uncommented the following line; however, this should generally by unnecessary
% and hints at improper citing.
%\nocite{*}
\tuhhbibliography{thesis}


% Appendix
% Feel free to add additional appendix chapters (e.g., measurement setups, etc.)
\begin{tuhhappendix}
  \chapter{Große Abbildungen}
Im Folgenden sind größere Abbildungen über den gesamten Speicher oder verfügbare Displayzeichen dargestellt.

\begin{tuhhtable}
  \newcommand{\display}[1]{\includegraphics[height=2ex]{images/chapter_5/display/#1.png}}
  \footnotesize\centering
  \begin{tabular}[tp]{rrcc   L{1mm}   rrcc}
%
  \THc{3}{c}{ASCII} & \THc{1}{c}{Kaffeevollautomat} &   \TRhc{1}{l}{}   &   \THc{3}{c}{ASCII} & \THc{1}{c}{Kaffeevollautomat} \\
  \THsub{1}{r}{hex} & \THsub{1}{r}{dez} & \THsub{1}{c}{Zeichen} & \THsub{1}{c}{Display} &   \TRhc{1}{l}{}   & \THsub{1}{r}{hex} & \THsub{1}{r}{dez} & \THsub{1}{c}{Zeichen} & \THsub{1}{c}{Display} \\
%
%  \abovebodyrule
%
       & {\tiny 0 – 31} & {\tiny$\ldots$} & {\tiny$\ldots$}   & \TRhc{1}{l}{} &   0x41 & 65            & A                          & A \\\TRc
  0x20 & 32             & Leerzeichen     & Leerzeichen       & \TRhc{1}{l}{} &   0x42 & 66            & B                          & B \\
  0x21 & 33             & !               & \_                & \TRhc{1}{l}{} &   0x43 & 67            & C                          & C \\\TRc
  0x22 & 34             & "               & \display{34}      & \TRhc{1}{l}{} &   0x44 & 68            & D                          & D \\
  0x23 & 35             & \#              & \display{35}      & \TRhc{1}{l}{} &   0x45 & 69            & E                          & E \\\TRc
  0x24 & 36             & \$              & \display{36}      & \TRhc{1}{l}{} &   0x46 & 70            & F                          & F \\
  0x25 & 37             & \%              & \display{37}      & \TRhc{1}{l}{} &   0x47 & 71            & G                          & G \\\TRc
  0x26 & 38             & \&              & \display{38}      & \TRhc{1}{l}{} &   0x48 & 72            & H                          & H \\
  0x27 & 39             & '               & \display{39}      & \TRhc{1}{l}{} &   0x49 & 73            & I                          & I \\\TRc
  0x28 & 40             & (               & \display{40}      & \TRhc{1}{l}{} &   0x4A & 74            & J                          & J \\
  0x29 & 41             & )               & \display{41}      & \TRhc{1}{l}{} &   0x4B & 75            & K                          & K \\\TRc
  0x2A & 42       & \textasteriskcentered & \display{42}      & \TRhc{1}{l}{} &   0x4C & 76            & L                          & L \\
  0x2B & 43             & +               & +                 & \TRhc{1}{l}{} &   0x4D & 77            & M                          & M \\\TRc
  0x2C & 44             & ,               & ,                 & \TRhc{1}{l}{} &   0x4E & 78            & N                          & N \\
  0x2D & 45             & -               & -                 & \TRhc{1}{l}{} &   0x4F & 79            & O                          & O \\\TRc
  0x2E & 46             & .               & .                 & \TRhc{1}{l}{} &   0x50 & 80            & P                          & P \\
  0x2F & 47             & /               & /                 & \TRhc{1}{l}{} &   0x51 & 81            & Q                          & Q \\\TRc
  0x30 & 48             & 0               & 0                 & \TRhc{1}{l}{} &   0x52 & 82            & R                          & R \\
  0x31 & 49             & 1               & 1                 & \TRhc{1}{l}{} &   0x53 & 83            & S                          & S \\\TRc
  0x32 & 50             & 2               & 2                 & \TRhc{1}{l}{} &   0x54 & 84            & T                          & T \\
  0x33 & 51             & 3               & 3                 & \TRhc{1}{l}{} &   0x55 & 85            & U                          & U \\\TRc
  0x34 & 52             & 4               & 4                 & \TRhc{1}{l}{} &   0x56 & 86            & V                          & V \\
  0x35 & 53             & 5               & 5                 & \TRhc{1}{l}{} &   0x57 & 87            & W                          & W \\\TRc
  0x36 & 54             & 6               & 6                 & \TRhc{1}{l}{} &   0x58 & 88            & X                          & X \\
  0x37 & 55             & 7               & 7                 & \TRhc{1}{l}{} &   0x59 & 89            & Y                          & Y \\\TRc
  0x38 & 56             & 8               & 8                 & \TRhc{1}{l}{} &   0x5A & 90            & Z                          & Z \\
  0x39 & 57             & 9               & 9                 & \TRhc{1}{l}{} &   0x5B & 91            & [                          & \display{91} \\\TRc
  0x3A & 58             & :               & :                 & \TRhc{1}{l}{} &   0x5C & 92            & \textbackslash             & \display{92} \\
  0x3B & 59             & ;               & ;                 & \TRhc{1}{l}{} &   0x5D & 93            & ]                          & \display{93} \\\TRc
  0x3C & 60             & <               & <                 & \TRhc{1}{l}{} &   0x5E & 94            & \^{}                       & \display{94} \\
  0x3D & 61             & =               & =                 & \TRhc{1}{l}{} &   0x5F & 95            & \_                         & \display{95} \\\TRc
  0x3E & 62             & >               & >                 & \TRhc{1}{l}{} &   0x60 & 96            & `                          & \display{96} \\
  0x3F & 63             & ?               & ?                 & \TRhc{1}{l}{} &        & {\tiny97-127} & {\tiny a-z \{ | \} $\sim$} & {\tiny$\ldots$} \\\TRc
  0x40 & 64             & @               & \display{64}      & \TRhc{1}{l}{} &        &               &                            &   \\
%
   \belowbodyrule
%
  \end{tabular}
  \caption{Verfügbarer Zeichensatz des Displays}
  \label{tbl:Displaysymbole}
\end{tuhhtable}

\begin{sidewaysfigure}
    \centering%
    \fboxsep=0.02\textwidth%
    \subfigure[EEPROM]{\label{subfig:EEPROM}\includegraphics[scale=0.7,trim={0 1.1cm 0 0},clip]{images/chapter_5/Speicher-Schema-Jura-EEPROM}}\\%
    \subfigure[RAM]{\label{subfig:RAM}\includegraphics[scale=0.7,trim={0 1.1cm 0 0},clip]{images/chapter_5/Speicher-Schema-Jura-RAM}}%
    \caption{Speicherschema der Jura Impressa S9}
    \label{fig:Speicherschema}
\end{sidewaysfigure}

\begin{sidewaystable}
  \footnotesize\centering
  \begin{tabular}[htp]{llllllllll}
  \THc{1}{l}{Funktion} &
  \THc{1}{c}{\rotatebox{50}{Maschine an}} &
  \THc{1}{c}{\rotatebox{50}{Schale fehlt}} &
  \THc{1}{c}{\rotatebox{50}{Schale leeren}} &
  \THc{1}{c}{\rotatebox{50}{Trester leeren}} &
  \THc{1}{c}{\rotatebox{50}{Wasser füllen}} &
  \THc{1}{c}{\rotatebox{50}{Gerät reinigen}} &
  \THc{1}{c}{\rotatebox{50}{Maschine spült}} &
  \THc{1}{c}{\rotatebox{50}{Abschließend spülen}} &
  \THc{1}{c}{\rotatebox{50}{Tassenbeleuchtung}} \\

%
  \TRhc{1}{l}{\textbf{Byte}} &
  \TRhc{1}{l}{\wort{03}} &
  \TRhc{1}{l}{\wort{0E}} &
  \TRhc{1}{l}{\wort{04}} &
  \TRhc{1}{l}{\wort{04}} &
  \TRhc{1}{l}{\wort{04}} &
  \TRhc{1}{l}{\wort{10}} &
  \TRhc{1}{l}{\wort{03}} &
  \TRhc{1}{l}{\wort{0D}} &
  \TRhc{1}{l}{\wort{0F}} \\
  
  Bit &
  \immer{\bitTrue{2}} &
  \immer{\bitTrue{2}} &
  \geteilt{\bitTrue{3}} &
  \bitTrue{5} &
  \geteilt{\bitTrue{3}} &
  \bitTrue{7} &
  \geteilt{\bitTrue{6,3,1}} &
  \bitTrue{3 oder 2} &
  \bitTrue{2 oder 1} \\
  
  \belowbodyrule
%
  \TRhc{1}{l}{\textbf{Byte}} &
  \TRhc{1}{l}{\wort{05}} &
  \TRhc{1}{l}{\wort{16}} &
  \TRhc{1}{l}{\wort{0E}} &
  \TRhc{1}{l}{\wort{0E}} &
  \TRhc{1}{l}{\wort{0E}} &
  \TRhc{1}{l}{\wort{22}} &
  \TRhc{1}{l}{\wort{13}} &
  \TRhc{1}{l}{} &
  \TRhc{1}{l}{} \\
  
  Bit &
  \immer{\bitTrue{5}} &
  \bitFalse{0} &
  \immer{\bitTrue{4}} &
  \bitTrue{5} &
  \immer{\bitTrue{6}} &
  \bitTrue{4} &
  \bitTrue{3,1} &
   &
   \\
  
  \belowbodyrule
%
  \TRhc{1}{l}{\textbf{Byte}} &
  \TRhc{1}{l}{\wort{16}} &
  \TRhc{1}{l}{\wort{29}} &
  \TRhc{1}{l}{\wort{1B}} &
  \TRhc{1}{l}{\wort{80}} &
  \TRhc{1}{l}{\wort{0F}} &
  \TRhc{1}{l}{} &
  \TRhc{1}{l}{\wort{17}} &
  \TRhc{1}{l}{} &
  \TRhc{1}{l}{} \\
  
  Bit &
  \immer{\bitTrue{1}} &
  \immer{\bitFalse{2}} &
  \immer{\bitTrue{1}} &
  \immer{\bitTrue{1,0}} &
  \immer{\bitFalse{4}} &
   &
  \bitTrue{1} &
   &
   \\
  
  \belowbodyrule
%
  \TRhc{1}{l}{\textbf{Byte}} &
  \TRhc{1}{l}{\wort{44}} &
  \TRhc{1}{l}{\wort{69}} &
  \TRhc{1}{l}{\wort{29}} &
  \TRhc{1}{l}{} &
  \TRhc{1}{l}{\wort{4C}} &
  \TRhc{1}{l}{} &
  \TRhc{1}{l}{\wort{62}} &
  \TRhc{1}{l}{} &
  \TRhc{1}{l}{} \\
  
  Bit &
  \immer{\bitTrue{0}} &
  \bitFalse{0} &
  \immer{\bitTrue{6-3}} &
   &
  \geteilt{\bitFalse{0}} &
   &
  \geteilt{\bitTrue{1}} &
   &
   \\
  
  \belowbodyrule
%
  \TRhc{1}{l}{\textbf{Byte}} &
  \TRhc{1}{l}{} &
  \TRhc{1}{l}{} &
  \TRhc{1}{l}{\wort{2B}} &
  \TRhc{1}{l}{} &
  \TRhc{1}{l}{\wort{91}} &
  \TRhc{1}{l}{} &
  \TRhc{1}{l}{} &
  \TRhc{1}{l}{} &
  \TRhc{1}{l}{} \\
  
  Bit &
   &
   &
  \immer{\bitTrue{6-3}} &
   &
  \geteilt{\bitFalse{1}} &
   &
   &
   &
   \\
  
  \belowbodyrule
%
  \TRhc{1}{l}{\textbf{Byte}} &
  \TRhc{1}{l}{\wort{1A}} &
  \TRhc{1}{l}{} &
  \TRhc{1}{l}{} &
  \TRhc{1}{l}{} &
  \TRhc{1}{l}{} &
  \TRhc{1}{l}{} &
  \TRhc{1}{l}{\wort{68}} &
  \TRhc{1}{l}{} &
  \TRhc{1}{l}{} \\
  
  Bit &
  \immer{\bitFalse{2}} &
   &
   &
   &
   &
   &
  \geteilt{\bitTrue{6}} &
   &
   \\
  
  Bit &
  \immer{\bitTrue{1,0}} &
   &
   &
   &
   &
   &
  \geteilt{\bitFalse{5,4}} &
   &
   \\
  \belowbodyrule
  \end{tabular}
  \caption{Speicherpositionen im \ac{RAM} (1)}
  \label{tbl:RAM1}
\end{sidewaystable}
\begin{sidewaystable}
  \footnotesize\centering
  \begin{tabular}[htp]{lllllllllll}
  \THc{1}{l}{Funktion} &
  \THc{1}{c}{\rotatebox{50}{Filter wechseln}} &
  \THc{1}{c}{\rotatebox{50}{Hahn offen}} &
  \THc{1}{c}{\rotatebox{50}{Teeportion}} &
  \THc{1}{c}{\rotatebox{50}{Dampfbezug}} &
  \THc{1}{c}{\rotatebox{50}{Wasserdampfportion}} &
  \THc{1}{c}{\rotatebox{50}{Pulver füllen}} &
  \THc{1}{c}{\rotatebox{50}{Bohnen füllen}} &
  \THc{3}{c}{\rotatebox{50}{Zubereitung}} \\
  
  \THsub{1}{l}{} &
  \THsub{1}{l}{} &
  \THsub{1}{l}{} &
  \THsub{1}{l}{} &
  \THsub{1}{l}{} &
  \THsub{1}{l}{} &
  \THsub{1}{l}{} &
  \THsub{1}{l}{} &
  \THsub{1}{c}{immer} &
  \THsub{1}{c}{1. Schritt} &
  \THsub{1}{c}{2. Schritt} \\

%
  \TRhc{1}{l}{\textbf{Byte}} &
  \TRhc{1}{l}{\wort{10}} &
  \TRhc{1}{l}{\wort{04}} &
  \TRhc{1}{l}{\wort{03}} &
  \TRhc{1}{l}{\wort{04}} &
  \TRhc{1}{l}{\wort{03}} &
  \TRhc{1}{l}{\wort{04}} &
  \TRhc{1}{l}{\wort{0E}} &
  \TRhc{1}{l}{} &
  \TRhc{1}{l}{} &
  \TRhc{1}{l}{\wort{03}} \\
  
  Bit &
  \bitTrue{5} &
  \geteilt{\bitTrue{3}} &
  \geteilt{\bitTrue{6}} &
  \geteilt{\bitTrue{3}} &
  \geteilt{\bitTrue{3}} &
  \bitTrue{0} &
  \bitTrue{7} &
   &
   &
  \bitTrue{1} \\
  
  \belowbodyrule
%
  \TRhc{1}{l}{\textbf{Byte}} &
  \TRhc{1}{l}{\wort{22}} &
  \TRhc{1}{l}{\wort{0F}} &
  \TRhc{1}{l}{\wort{0B}} &
  \TRhc{1}{l}{\wort{0B}} &
  \TRhc{1}{l}{\wort{0B}} &
  \TRhc{1}{l}{} &
  \TRhc{1}{l}{} &
  \TRhc{1}{l}{} &
  \TRhc{1}{l}{} &
  \TRhc{1}{l}{\wort{0B}} \\
  
  Bit &
  \bitTrue{3} &
  \immer{\bitFalse{6}} &
  \geteilt{\bitFalse{3}} &
  \geteilt{\bitFalse{3}} &
  \geteilt{\bitFalse{3}} &
   &
   &
   &
   &
  \geteilt{\bitFalse{3}} \\
    
  \belowbodyrule
%
  \TRhc{1}{l}{\textbf{Byte}} &
  \TRhc{1}{l}{\wort{F8}} &
  \TRhc{1}{l}{\wort{4C}} &
  \TRhc{1}{l}{\wort{0F}} &
  \TRhc{1}{l}{\wort{13}} &
  \TRhc{1}{l}{\wort{13}} &
  \TRhc{1}{l}{} &
  \TRhc{1}{l}{} &
  \TRhc{1}{l}{} &
  \TRhc{1}{l}{} &
  \TRhc{1}{l}{\wort{17}} \\
  
  Bit &
  \immer{\bitTrue{0}} &
  \geteilt{\bitFalse{0}} &
  \immer{\bitTrue{5}} &
  \bitTrue{2-0} &
  \bitTrue{2,0} &
   &
   &
   &
   &
  \bitTrue{3} \\
  
  \belowbodyrule
%
  \TRhc{1}{l}{\textbf{Byte}} &
  \TRhc{1}{l}{} &
  \TRhc{1}{l}{\wort{4D}} &
  \TRhc{1}{l}{\wort{4C}} &
  \TRhc{1}{l}{\wort{49}} &
  \TRhc{1}{l}{\wort{49}} &
  \TRhc{1}{l}{} &
  \TRhc{1}{l}{} &
  \TRhc{1}{l}{} &
  \TRhc{1}{l}{} &
  \TRhc{1}{l}{\wort{62}} \\
  
  Bit &
   &
  \bitTrue{3,1} &
  \geteilt{\bitFalse{0}} &
  \geteilt{\bitTrue{0}} &
  \geteilt{\bitTrue{0}} &
   &
   &
   &
   &
  \geteilt{\bitTrue{1}} \\
  
  \belowbodyrule
%
  \TRhc{1}{l}{\textbf{Byte}} &
  \TRhc{1}{l}{} &
  \TRhc{1}{l}{} &
  \TRhc{1}{l}{\wort{68}} &
  \TRhc{1}{l}{\wort{4C}} &
  \TRhc{1}{l}{\wort{4C}} &
  \TRhc{1}{l}{} &
  \TRhc{1}{l}{} &
  \TRhc{1}{l}{} &
  \TRhc{1}{l}{} &
  \TRhc{1}{l}{\wort{68}} \\
  
  Bit &
   &
   &
  \geteilt{\bitTrue{6}} &
  \geteilt{\bitFalse{0}} &
  \geteilt{\bitFalse{0}} &
   &
   &
   &
   &
  \geteilt{\bitFalse{5}} \\
  
  \belowbodyrule
%
  \TRhc{1}{l}{\textbf{Byte}} &
  \TRhc{1}{l}{\wort{F9}} &
  \TRhc{1}{l}{} &
  \TRhc{1}{l}{\wort{4D}} &
  \TRhc{1}{l}{} &
  \TRhc{1}{l}{} &
  \TRhc{1}{l}{} &
  \TRhc{1}{l}{} &
  \TRhc{1}{l}{\wort{0A}} &
  \TRhc{1}{l}{\wort{03}} &
  \TRhc{1}{l}{} \\
  
  Bit &
  \bitTrue{7,6,5,4,2} &
   &
  \bitFalse{3,1} &
   &
   &
   &
   &
  \bitTrue{6} &
  \bitTrue{7,4} &
   \\
  
   &
  (\wert{$0_{10} \rightarrow 244_{10}$}) &
   &
  \bitTrue{0} &
   &
   &
   &
   &
  (2-0: "`Kaffee"') &
  (7 nicht manuell) &
   \\
  \belowbodyrule
  \end{tabular}
  \caption{Speicherpositionen im \ac{RAM} (2)}
  \label{tbl:RAM2}
\end{sidewaystable}

\begin{sidewaysfigure}
  \begin{center}
    \includegraphics[scale=0.94]{images/chapter_5/API-EEPROM}
    \caption{Abgefragte Speicherstellen für die \ac{API} im \ac{EEPROM} (1 Kästchen $\equiv$ 1 Wort)}
    \label{fig:API-EEPROM}
  \end{center}
\end{sidewaysfigure}
\begin{sidewaysfigure}
  \begin{center}
    \includegraphics[scale=1]{images/chapter_5/API-RAM}
    \caption{Abgefragte Speicherstellen für die \ac{API} im \ac{RAM} (1 Kästchen $\equiv$ 1 Byte)}
    \label{fig:API-RAM}
  \end{center}
\end{sidewaysfigure}

  \chapter{Inhalt der CD}

\begin{tuhhtable}
  \footnotesize\centering
  \begin{tabular}[tb]{L{.35\textwidth}L{.55\textwidth}}
    \THc{1}{c}{Inhalt}                    & \THc{1}{c}{Beschreibung} \\
    \abovebodyrule
    thesis.pdf                            & Diese Arbeit in digitaler Form \\\TRc
    JuraCoffeeMemory/                     & Das entwickelte Projekt \\
    JuraCoffeeMemory/arduino/arduino.ino  & Das Arduino Skript aus \cite{GitCoffeeMachine} \\\TRc
    JuraCoffeeMemory/data/                & Die aufgenommenen Speicherauszüge im JSON Format \\
    JuraCoffeeMemory/doxygen/             & Die Doxygen Dokumentation \\\TRc
    JuraCoffeeMemory/manual\_jura/        & Handbuch und Abbildung des Kaffeevollautomaten aus \cite{GitCoffeeMachine} \\
    JuraCoffeeMemory/result/              & Ergebnisse der Auswertung \\\TRc
    JuraCoffeeMemory/tools/               & Kleine Helfer-Programme und Codestücke \\
    JuraCoffeeMemory/website/             & Die Webseite zur Visualisierung \\\TRc
    \belowbodyrule
  \end{tabular}
  \caption{Inhalt der CD}
  \label{tbl:elements}
\end{tuhhtable}

\end{tuhhappendix}


% The End
\end{document}
