\chapter{Einführung}
\todo Einleitungssätze.

\section{Motivation}
Heutige Geräte versprechen viel Komfort und eine einfache Handhabung. Über das Internet werden die Geräte zunehmend vernetzt und smart.
Diese Arbeit befasst sich mit einer Kaffeemaschine und in erster Linie mit ihrem Speicher. Dieser merkt sich nicht nur Betriebszustände,
sondern auch Einstellungen, Zählstände und evtl. vieles mehr. Diese Informationen können nicht nur gelesen, sondern zum Teil auch verändert werden.
Moderne Funktionen stecken ohne eine entsprechende graphische Schnittstelle aber bereits in älteren Maschinen, unter der Voraussetzung über das entsprechende Wissen zu verfügen.

\todo Deutlich mehr ausholen: Was ist IoT? Warum braucht man das? Warum ist das relevant für diese Arbeit? Und dann zur Aufgabenstellung überleiten.

\section{Aufgabenstellung}
Anhand eines \textit{Jura Impressa S9 Kaffeevollautomaten} soll der Speicher untersucht werden.
Welches Wort / Byte / Bit speichert welche Information? Welche Bedeutung haben diese auf den Betrieb?
Hierfür sollen Skripte erstellt werden und der Speicher systematisch untersucht werden.

Wenn die Denkweise der Kaffeemaschine bekannt ist, sollen Werte im \ac{EEPROM} abgefragt, aber auch gezielt verändert werden, sowie Statusinformationen aus dem \ac{RAM} ausgelesen werden.
Die erhaltenen Rohinformationen werden nach den gewonnen Erkenntnissen aufbereitet und stehen so für weitere Projekte zur Verfügung.

Als kleine Demonstrationen entsteht am Ende ein Programm, welches Profile anlegen kann, sodass jeder Nutzer auf Knopfdruck einen Kaffee nach seinen Lieblings Präferenzen zubereitet bekommt.

\todo Aufbau der Arbeit beschreiben: In Kapitel 2 werden grundlegende Begriffe erklärt und ein Einblick in den Stand der Technik gegeben...
